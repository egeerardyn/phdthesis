% !TEX root =  LPMSmooth_ML_EG_JS_JL.tex

\subsection{Obtaining the Impulse Response Function}

The estimated impulse response function  $\hat g_{\mathrm{poly}\setminus \mathrm{DC
}}(t)$ is obtained explicitly via the IDFT of the LPM-estimate $\hat G_\mathrm{poly}(\Omega_k)$ of the FRF in accordance with equation \eqref{eq:defDFT}, \emph{viz}:

\begin{align}\label{eq:impRespiDFT}
\hat g_{\mathrm{poly}\setminus \mathrm{DC
}}(t) &= \frac{1}{\sqrt{N}}\sum_{k=1}^{N-1}\hat G_\mathrm{poly}(\Omega_k)e^{\frac{j2\pi kt}{N}}
\end{align}

%\JL{
\noindent where the estimated FRF in the frequency band between  the Nyquist and the sample frequencies is obtained as follows:

\begin{align}
\hat G_\mathrm{poly}(\Omega_{N-k}) = \overline{\hat G_\mathrm{poly}(\Omega_k)},\ \text{for}\ k=1,\dots,N/2
\end{align}

\noindent 
(with $\overline{\hat G_\mathrm{poly}}$ the complex conjugate of $\hat G_\mathrm{poly}$)
to ensure $\hat g_\mathrm{poly}(t)$ to be real.
%}

Smoothing of the estimated FRF requires the correct estimate of the impulse response. Unfortunately, the LPM presented in Section II and in \cite{schoukens2010nonparametric} does not estimate the FRF  at the frequency $\Omega_0$ (i.e. DC value of the FRF), hence the subscript $\setminus\mathrm{DC}$ in equation \eqref{eq:impRespiDFT}. Consequently, the mean value of the corresponding estimated impulse response given in equation \eqref{eq:impRespiDFT} is not correct. This anomaly is rectified by developing a simple estimator of the mean value, presented below.
