
\subsubsection{Truncation Via a Statistical Test}\label{se:truncstattest}

\newcommand{\segm}[2]{\ensuremath{{#1}^{[#2]}}}

To distinguish the part of the measured impulse response where the actual impulse response dominates from the part where the noise dominates, the estimated impulse response $\hat{g}_\mathrm{poly}$ is split into smaller segments. 
One can then gradually determine whether these segments are noise or the actual impulse response.

% 1) cut into NS segments of minimally NM samples
% 2) assume last segment == noise
%    start from back
% 3) F-test on Var, Var==RMS if dc == 0 (i.e. noise)
%    H0: s²(1) =  s²(2)
%    H1: s²(1) >  s²(2)
%    alpha >> ==> beta <<
% 4) if H1 and RMS(signal)/RMS(noise) > f
%    robustness, SNR
%    select segment i
% 5) refine until wanted resolution is obtained

\begin{enumerate}
  \item The impulse response $\hat{g}_\mathrm{poly}(t)$ is split into $N_S$ segments of approximate equal lengths
$L_S = \floor{\frac{N}{N_S}}$. 
If necessary, the length of the first segment is reduced to allow for the others to have equal lengths.
The $i^{\text{th}}$ segment is denoted $\segm{\hat{g}}{i}$, for $i \in \left\{1,\ldots,N_S\right\}$.
  By \assref{ass:decay90perctime}, the last segment, $\segm{\hat{g}}{N_S}$, is noise when $L_S \leqslant 0.1N$.
  
  \item Iteration is then carried out from the last segment ($i = N_S$) towards the beginning of the impulse response. To determine whether the preceding segment $\segm{\hat{g}}{i-1}$ is completely dominated by noise,  both an $F$-test and an absolute criterion on the RMS are used.

  The $F$-test is suited to compare the variances of different segments $\segm{\hat{g}}{i}$ and $\segm{\hat{g}}{i-1}$ \citep{Parsons1974}, while accounting for the degrees of freedom in each segment.
  For zero-mean noise, these variances can be related to the RMS value of each noise segment.
  This has the advantage that a segment in which the impulse response still has a significant amplitude is likely to have a large RMS value, which is more likely to be detected.
  Let $s^2(i)$ denote the mean square value of segment $\segm{\hat{g}}{i}$, \emph{viz}:
  \begin{equation}
    s^2(i) 
    \isdef \frac{1}{L_S} \sum_{j=1}^{L_S} \left(\segm{\hat{g}}{i}(j)\right)^2
    = \left( \rms{\hat{g}^{[i]}} \right)^2
           \text{.}
  \end{equation}
  Starting with the last segment ($i = N_S$), the following null hypothesis and the alternative hypothesis are formulated:
  \begin{align}
     \nullHypothesis &: s^2(i-1) = s^2(i)\\
     \altHypothesis    &: s^2(i-1) > s^2(i)
     \text{.}
  \end{align}
  As segment $\segm{\hat{g}}{i}$ has been tested and been classified as noise, the null hypothesis states that the segment $\segm{\hat{g}}{i-1}$ is also noise.
  The alternative case occurs when $\segm{\hat{g}}{i-1}$ has a significant change, which indicates the presence of the signal.
  Using the $F$-test on the RMS values of the two segments with a high confidence level of $\alpha=0.99$ and the aforementioned hypotheses, we can determine whether $\segm{\hat{g}}{i-1}$ is likely to be part of the actual impulse response.
  The high level of confidence ($\alpha=0.99$) ensures that the probability of a Type~II error is smaller than $1-\alpha$ \citep{Parsons1974}.
  In our case, such an error means that a part of the actual impulse response would falsely be regarded as noise, which could significantly increase the bias as information of the system is discarded.
  A Type~I error is less detrimental since, in that case, noise is falsely classified as a signal component and kept in the impulse response, thereby causing a sub-optimal value of the variance of the estimate.
  As the LPM samples are correlated over a short frequency span, the actual noise present in $\hat{g}$ may be slightly non-stationary.
  To cope with this, one can introduce other criteria which must be satisfied together with the outcome of the $F$-test.
  A criterion that shows good results is to check whether the segment $\segm{\hat{g}}{i-1}$ has an RMS value that is at least a factor $\kappa$ larger than the RMS of the noise.
  Even for a moderate $\kappa = 1.1$, a large improvement in the detection was observed. 
%  \JL{Q: doesn't this increase the probability of a Type II error?}
%  \EG{A: This slightly increases for small $\kappa$ and increases significantly for large values of $\kappa$. But it is needed to overcome nonstationarity of the noise.}

  \item This procedure is repeated until a segment $\segm{\hat{g}}{i-1}$ is found that is dominated by the actual impulse response according to the outcome of the $F$-test and the absolute RMS value of the segment.
  At that point, it is very likely that the transition from noise to the actual impulse response happens within the segment $\segm{\hat{g}}{i-1}$.
  One now has the choice to accept the last sample of $\segm{\hat{g}}{i-1}$ to be the last meaningful sample $t_{\mathrm{trunc}}$ for the impulse response.
  The accuracy of this estimate is limited by the length of the segment $L_S$.

  \item
  A more accurate estimate can be obtained by dividing the segment $\segm{\hat{g}}{i-1}$ -- which may contain both the signal and noise -- yet again into smaller segments. The procedure described above is then repeated until a satisfactory, accurate $t_{\mathrm{trunc}}$ is obtained, or until the subsegments have a length that is too short ($L_{S\min} = 10$) to guarantee the accuracy of the RMS values. %rely on the RMS values to be accurate.
  To start this refinement, the last $\segm{\hat{g}}{i}$ should be used to compare the RMS with the $F$-test, since the last subsegment of $\segm{\hat{g}}{i-1}$ cannot be asserted to be dominated by noise.

\end{enumerate}
This procedure is illustrated in \figref{FRF_truncate_stat_EG} for the system described by the following simulation equations, which are relevant to the LPM outlined in Section \ref{se:LPMFRFest}:
\begin{subequations}
\label{eq:systemSimulations}
\begin{align}
y_0(t)  &= 1.5371y_0(t-1)    -0.9025y_0(t-2) + u(t)\\
y(t)       &= y_0(t) + e(t),
\end{align}
\end{subequations}
where $e(t)$ is a white noise sequence, such that the SNR of the output signal is 78.2~dB.
% \eqref{eq:systemSimulations}. 
The figure shows segments of the estimated impulse response. The last two segments are dominated by the noise, while all samples at $t<80$ have large system contributions. The algorithm outlined above selects $t_{\mathrm{trunc}} = 80$, beyond which all samples are set to zero, resulting in a smooth estimate.  %after which all samples are fixed to a zero value for the smoothed estimate.

\begin{figure}[tbh]
\centering
 \setlength\figureheight{0.6\columnwidth}
 \setlength\figurewidth{0.8\columnwidth}
  % This file was created by matlab2tikz v0.2.2.
% Copyright (c) 2008--2012, Nico Schlömer <nico.schloemer@gmail.com>
% All rights reserved.
%
% The latest updates can be retrieved from
%   http://www.mathworks.com/matlabcentral/fileexchange/22022-matlab2tikz
% where you can also make suggestions and rate matlab2tikz.
%
%
%
\begin{tikzpicture}

\begin{axis}[%
width={\figurewidth},
height={\figureheight},
scale only axis,
xmin=0, xmax=160,
xtick={0,40,120,160},
extra x ticks={80},
extra x tick labels={$\truncTime$},
xlabel={Time $t$ \axisunit{samples}},
ymin=-1.3, ymax=1.7,
ylabel={Impulse Response $g$ \noaxisunit},
xmajorgrids,
legend style={nodes=right},
unbounded coords=jump]

\addplot [G0Hat] table[]{\thisDir/data/truncRMS/true.tsv};
\addlegendentry{True Value}

\addplot [color=lightgray!80!black,solid] table[]{\thisDir/data/truncRMS/full.tsv};
\addlegendentry{$\estimated g$}
 
\addplot [color=black,solid,line width=1pt] table[]{\thisDir/data/truncRMS/truncated.tsv};
\addlegendentry{$\estimated g_{\trunc}$}

\addplot [rmsOfSegment] table[]{\thisDir/data/truncRMS/RMS.tsv};
\addlegendentry{$\mathrm{RMS} (\segm{\estimated g}{i})$}

\addplot [truncationline] table[]{\thisDir/data/truncRMS/truncationLine.tsv};

\node[annotation] at (axis cs: 20,-1.1) {Segment 1};
\node[annotation] at (axis cs: 60,-1.1) {Segment 2};
\node[annotation] at (axis cs: 100,-1.1) {Segment 3};
\node[annotation] at (axis cs: 140,-1.1) {Segment 4}; 

\end{axis}


\end{tikzpicture}%

\caption{Schematic of statistical determination of a smooth (improved) FRF. 
The different segments are separated by dotted lines.}
\label{FRF_truncate_stat_EG}
\end{figure}
