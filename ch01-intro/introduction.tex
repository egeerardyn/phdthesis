\chapter{Introduction}
\def\thisDir{ch01-intro}
\tikzsetfigurename{ch01fig}
\myEpigraph{It's a dangerous business, Frodo, going out your door. You step onto the road, and if you don't keep your feet, there's no knowing where you might be swept off to.}{John Ronald Reuel Tolkien}{The Fellowship of the Ring}

In this chapter, we first discuss some basic aspects of system identification, some typical choices and how those relate to user-friendly system identification and this dissertation.

\section{System Identification}

System identification is the science that deals with converting observations of a system into mathematical models to describe the behavior of the system under test.
Successfully identifying a system often hinges on a plethora of choices that are to be made by the user.
As a result, system identification can also be seen as an art for which the practitioner needs to learn the tricks of the trade.
In \figref{fig:intro:identification-cycle}, a high-level flow diagram containing some key steps that are performed when identifying a system.
These are the following steps:

\paragraph{Experiment design}
In the first step, an experiment is constructed to gain insight into the behavior of a system.
For system identification, experiment design is a question of designing input signals that will be applied to the system under test.
Typically, the design consists of constructing a signal such that the obtained model  has a small uncertainty in some sense~\citep{Goodwin1977,Goodwin2006GBO}.
On the one hand, if one has very precise prior information, one could even construct optimal signals~\citep{Narasimhan2011}.
Typically, the design is very sensitive to the prior knowledge of the system.
This is necessary to focus all the energy in the signal towards exciting the system dynamics as to reduce the uncertainty in the model.
However, when the prior knowledge is incomplete, uncertain or even incorrect, optimal signals can lead to very poor models as they might not excite other dynamics.
An alternative approach is to incorporate the uncertainty of the prior knowledge into the signal design.
This leads to so-called robust optimal experiment design~\citep{Rojas2012,Goodwin2006} that construct signals that optimize the worst-case model uncertainty.
Paradoxically, this means that for robust optimal designs, more prior knowledge, rather than less is required.

The design of good experiments hence already forces the user to make a choice that can have far-reaching consequences for the model quality.

\paragraph{Select a model structure}
Once an experiment has been performed, the obtained input-output data is to be processed.
In many cases, the eventual goal of identification is to obtain a parametric model, so the user has to select a model structure and/or complexity based on both his prior knowledge and on the data.
A useful tool to help choose the model structure and/or model orders, is to inspect a nonparametric model of the system.
One such tool is the frequency response function.
For instance, by inspecting a frequency response function, a control engineer can have a rough idea of the system behavior in term of  approximate resonance frequencies, asymptotic slopes, \ldots
This gives a qualitative view of the minimally required complexity of the model and is hence a useful diagnostic tool.
However, since most parametric estimators enforce very little prior knowledge about the system, noise in the measurements can sometimes be a problem to distinguish the actual system features.

\paragraph{Estimate the model}
Once the model structure has been chosen, its parameters are to be estimated.
This is done by formulating an optimization problem which serves to minimize a particular cost function.
The cost function is in essence a measure of the differences between the measured data and the model evaluated for particular parameter values.
By minimizing the cost function, the optimization procedure yields parameter values that try to minimize the misfit between the data and the model.
Unfortunately, for many identification settings, these optimization problems are non-convex~\citep{Boyd2004}.
This means that initial values need to be available to start the optimization process.
Moreover, for poor starting values, optimization algorithms may get stuck in local minima of the cost function.
Consequently, without reasonable starting values, the `optimized' model may also be of poor quality. 

\paragraph{Validate the model}
Finally, the estimated model is validated.
Essentially, one checks whether the obtained model meets the predefined requirements, e.g. with respect to allowable uncertainty, compexity, \ldots
This can happen in a few different ways.
On the one hand, in-sample measures such as statistical tests on the value of the cost function, correlation tests on the residuals, \ldots allow to verify that the model performs well on the measured dataset.
Such approaches help to detect underfitting, where the model is not flexible enough to describe the measurements.
On the other hand, the ultimate test is to validate the model against new measurements that have not been used during the estimation.
This prevents overfitting of the model, where the model not only captures the behavior of the system but also describes the particular realization of the noise in the measurements.
The latter leads to models that are overly tuned to the measurements such that their results cannot be replicated for new measurements.
In case the model is not validated, this provides the user with new knowledge: that particular model is not good enough and at least one of the previous steps (or the prior knowledge) has to be altered.


\begin{figure}
  \centering
  \includegraphics[width=0.9\columnwidth]{\thisDir/figs/id-cycle.pdf}
  \caption[System identification loop]{System identification loop. \disclaimer{Adapted from \citep[Figure 1.10]{Ljung1999} and \citep{Mehra1981}.}}
  \label{fig:intro:identification-cycle}
\end{figure}

\subsection{System Identification Approaches}
In this section, some different approaches for system identification are discussed and related to the work in this dissertation.
\subsubsection{White Box, Black Box or Grey Box}


\subsubsection{Time or Frequency Domain}
\subsubsection{Linear or nonlinear?}
\subsubsection{Time invariant or not?}
\subsubsection{Parametric or non-parametric}

\section{User-friendly System Identification}
In this dissertation, we focus on developing system identification techniques that are `user-friendly'.
User-friendly should be understood as having a good user experience for two groups of users:
\begin{itemize}
  \item novice users, without formal training in system identification, optimization, \ldots and,
  \item well-seasoned identification practitioners that are already able to build good models.
\end{itemize}
Concretely, for novice users it is important to have straightforward techniques that require little interaction.
In practice, this boils down to methods that work well for a wide class of systems.
As such, only very generic assumptions of the system under test should be made and these should be easy to interpret.

For seasoned identification practitioners, methods that require little interaction are an opportunity for automation.
This is especially important for complex systems: systems that have high-order dynamics, \gls{MIMO} systems with a high dimensionality, \ldots
For such complex systems, building a model can be time-consuming and laborious if one has to supervise every step of the process.
For more advanced users, user-friendly methods can hence allow to deal with more complex systems in a shorter amount of time such that the economical cost of building a model is reduced.

In this dissertation, the focus lies on \gls{LTI} systems.
While this might seem as a very restrictive choice, this is one of the fundamental settings for system identification.
In particular, \gls{LTI} systems are a first step to build a system model.
In many cases, such a linear model is accurate enough, e.g. to build a nominal controller, to design electrical filters and obtain a reasonable intuition of the system under test.
Also, most engineers and scientists have a good understanding of linear systems such that \gls{LTI} models align well with their prior knowledge and experience.
Nevertheless, when a linear model is not adequate, a more flexible model needs to be constructed.
Such a model could be more flexible, e.g. by relinquishing the time-invariance and/or its linearity.
However, many of those advanced modeling approaches employ \gls{LTI} models as starting values or intermediate result in building the actual non-linear~\citep{Giri2010} or time-varying model~\citep{Lataire2012,Louarroudi2014}.
As such, improvements in estimating \gls{LTI} models also indirectly improve more advanced methods.

\subsection{Contributions}
In this dissertation, we look into a few aspects of the identification workflow with the goal to make the whole process more user-friendly.

\paragraph{How should a good experiment be designed?}
Can we design a `good experiment' to identify a \gls{LTI} model?
Particularly, this means a robust input signal needs to be constructed without relying on extensive prior knowledge of the system.
Such a signal should cover a wide frequency band to excite all dynamics of the system during the experiment.
However, the signal should ensure that systems in different frequency bands can be identified with a specified level of accuracy.

\paragraph{Can we process the input/output data in a non-parametric way that offers more insight than a standard frequency response function?}
Developing a full parametric model for complicated systems requires considerable effort.
Instead, could we leverage non-parametric or locally parametric approaches to obtain insight in the behavior of the system?
In particular, can local modeling methods be used to characterize the resonances of a system, or even work around the typical frequency resolution limitations of a discrete frequency grid?

\paragraph{Can we avoid local minima by using smoothers?}
When a parametric model is required, this often involves solving a nonlinear optimization problem.
Such optimization problems require good initial parameter estimates to allow iterative methods to converge to a good local optimum, or, preferably even the global optimum.
Can non-parametric smoothers be used to help avoid such local optima?


\section{Outline and Publications}
   The lion's share of this thesis has been published in either peer-reviewed  journals or conferences.
   This section links my different publications to the different sections in this thesis.
   For an overview of publications grouped by type, please refer to page~\pageref{publicationList}.

\begin{refsection}
% http://tex.stackexchange.com/questions/38580/displaying-selected-bibliographic-items-in-the-body-of-the-text
% http://tex.stackexchange.com/questions/126226/how-do-i-instruct-fullcite-to-use-maxbibnames-rather-than-maxcitenames

\makeatletter
\DeclareCiteCommand{\fullcite}
  {\defcounter{maxnames}{\blx@maxbibnames}%
    \usebibmacro{prenote}}
  {\usedriver
     {\DeclareNameAlias{sortname}{default}}
     {\thefield{entrytype}}}
  {\multicitedelim}
  {\usebibmacro{postnote}}
\DeclareCiteCommand{\footfullcite}[\mkbibfootnote]
  {\defcounter{maxnames}{\blx@maxbibnames}%
    \usebibmacro{prenote}}
  {\usedriver
     {\DeclareNameAlias{sortname}{default}}
     {\thefield{entrytype}}}
  {\multicitedelim}
  {\usebibmacro{postnote}}
\makeatother

% http://tex.stackexchange.com/questions/18664/underline-my-name-in-the-bibliography
% Plus see release notes of BIBLATEX 3.3/3.4
\DeclareNameFormat{author}{%
  \nameparts{#1}%
\ifthenelse{\equal{\namepartfamily}{Geerardyn}}%
    {\textbf{\ifblank{\namepartgiven}{}{\namepartgiven\space}\namepartfamily}}%
    {\ifblank{\namepartgiven}{}{\namepartgiven\space}\ifblank{\namepartprefix}{}{\namepartprefix\space}\namepartfamily}%
\ifthenelse{\value{listcount}<\value{liststop}}%
    {\addcomma\space}
    {}}


The quasi-logarithmic multisines presented in \chapref{sec:excitation} are based on a journal article published in \gls{IEEE} Transactions on Instrumentation \& Measurement:
\begin{itemize}
  \item \fullcite{Geerardyn2013TIM}, 
\end{itemize}
preliminary results were also presented at the 2012 \gls{IFAC} symposium on System Identification (\textsc{SYSID}) and the \gls{IEEE} International Instrumentation and Measurement Conference (\textsc{I$^{\text{2}}$MTC}):
\begin{itemize}
  \item \fullcite{Geerardyn2012IMTC}, and
  \item \fullcite{Larsson2012SYSID}.
\end{itemize}
This work has also been presented at the following local (non-refereed) conferences:
\begin{itemize}
  \item \fullcite{Geerardyn2012Benelux}, and
  \item \fullcite{Geerardyn2012ERNSI}.
\end{itemize}

The non-parametric estimation methods in \chapref{sec:nonparametric} are based on a yet-unpublished manuscript.
 The so-called time-truncated \gls{LPM} that is presented in the same chapter is based on a journal published in \gls{IEEE} Transactions on Instrumentation \& Measurement:
\begin{itemize}
  \item \fullcite{Lumori2014TIM}.
\end{itemize}
In particular, this smoother enables one to reduce the effect of noise on an estimated \gls{FRF} using an automated approach.
Relatedly, a preliminary study of the \gls{LPM} in the context of lightly-damped \gls{MIMO} systems has been presented at the (non-refereed) Benelux Meeting on Systems and Control:
\begin{itemize}
    \item \fullcite{Verbeke2015Benelux}.
\end{itemize}

The use of the non-parametric \gls{FRF} estimation methods for \Hinf gain estimation (and in general the \gls{FRF} interpolation from \chapref{sec:hinf}) has been presented at the 2014 \gls{IFAC} World Conference in South Africa and the 2014 Leuven Conference on Noise and Vibration Engineering (ISMA) in Leuven:
\begin{itemize}
  \item \fullcite{Geerardyn2014IFAC},
  \item \fullcite{Geerardyn2014ISMA}.
\end{itemize}
Preliminary results have been presented at local (non-refereed) conferences:
\begin{itemize}
  \item \fullcite{Geerardyn2013Benelux},
  \item \fullcite{Geerardyn2013ERNSI},
  \item \fullcite{Geerardyn2014Benelux},
  \item \fullcite{Geerardyn2014DYSCO}, and
  \item \fullcite{Geerardyn2014ERNSI}.
\end{itemize}
Experimental work related to \chapref{sec:hinf} has also been presented at the 2015 \gls{IFAC} Symposium on System Identification (\textsc{sysid}) and :
\begin{itemize}
  \item \fullcite{Voorhoeve2015SYSID}
  \item \fullcite{Voorhoeve2015ERNSI}
\end{itemize}

The study of different initialization strategies as depicted in \chapref{sec:initvals} has been published in the \gls{IEEE} Transactions on Instrumentation \& Measurement:
\begin{itemize}
  \item \fullcite{Geerardyn2015TIM},
\end{itemize}
and also at the local (non-refereed) 2015 Benelux Meeting on Systems and Control:
\begin{itemize}
    \item \fullcite{Geerardyn2015Benelux}.
\end{itemize}

In collaboration with fellow researchers, I have written a few other publications.
However, these publications are not covered in this dissertation.

\begin{itemize}
  \item \fullcite{vanBerkel2014Automatica},
  \item \fullcite{Cooman2012SMACD},
  \item \fullcite{Cooman2012DYSCO},
  \item \fullcite{Cooman2012ERNSI}.
\end{itemize}
\end{refsection}

