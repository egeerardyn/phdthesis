\chapter{Introduction}
\def\thisDir{ch01-intro}
\tikzsetfigurename{ch01fig}
\myEpigraph{It's a dangerous business, Frodo, going out your door. You step onto the road, and if you don't keep your feet, there's no knowing where you might be swept off to.}{John Ronald Reuel Tolkien}{The Fellowship of the Ring}

In this chapter, we first discuss some basic aspects of system identification, some typical choices and how those relate to user-friendly system identification and this dissertation.

\section{System Idenfication}

System identification is the scientific approach of converting observations of a system into mathematical models that describe the behavior of the system under test.
In \figref{fig:intro:identification-cycle}, a high-level flow diagram containing some key steps that are performed when identifying a system.
Note that this is a simplified view, different authors 

\begin{figure}
  \centering
  \includegraphics[width=0.9\columnwidth]{\thisDir/figs/id-cycle.pdf}
  \caption[System identification loop]{System identification loop. \disclaimer{Adapted from \citep[Figure 1.10]{Ljung1999} and \citep{Mehra1981}.}}
  \label{fig:intro:identification-cycle}
\end{figure}

\subsection{Linear System theory}
\subsection{System Identification Approaches}

\subsubsection{White Box, Black Box or Grey Box}
\subsubsection{Time or Frequency Domain}
\subsubsection{Linear or nonlinear?}
\subsubsection{Time invariant or non?}

\section{User-friendly System Identification}
In this dissertation, we focus on developing system identification techniques that are `user-friendly'.
User-friendly should be understood as having a good user experience for two groups of users:
\begin{itemize}
  \item novice users, without formal training in system identification, optimization, \ldots and,
  \item well-seasoned identification practitioners that are already able to build good models.
\end{itemize}
Concretely, for novice users it is important to have straightforward techniques that require little interaction.
In practice, this boils down to methods that work well for a wide class of systems.
As such, only very generic assumptions of the system under test should be made and these should be easy to interpret.

For seasoned identification practitioners, methods that require little interaction are an opportunity for automation.
This is especially important for complex systems: systems that have high-order dynamics, \gls{MIMO} systems with a high dimensionality, \ldots
For such complex systems, building a model can be time-consuming and laborious if one has to supervise every step of the process.
For more advanced users, user-friendly methods can hence allow to deal with more complex systems in a shorter amount of time such that the economical cost of building a model is reduced.

In this dissertation, the focus lies on \gls{LTI} systems.
While this might seem as a very restrictive choice, this is one of the fundamental settings for system identification.
In particular, \gls{LTI} systems are a first step to build a system model.
In many cases, such a linear model is accurate enough, e.g. to build a nominal controller, to design electrical filters and obtain a reasonable intuition of the system under test.
Also, most engineers and scientists have a good understanding of linear systems such that \gls{LTI} models align well with their prior knowledge and experience.
Nevertheless, when a linear model is not adequate, a more flexible model needs to be constructed.
Such a model could be more flexible, e.g. by relinquishing the time-invariance and/or its linearity.
However, many of those advanced modeling approaches employ \gls{LTI} models as starting values or intermediate result in building the actual non-linear~\citep{Giri2010} or time-varying model~\citep{Lataire2012,Louarroudi2014}.
As such, improvements in estimating \gls{LTI} models also indirectly improve more advanced methods.

\subsection{Contributions}
In this dissertation, we look into a few aspects of the identification workflow with the goal to make the whole process more user-friendly.

\paragraph{How should a good experiment be designed?}
Can we design a `good experiment' to identify a \gls{LTI} model?
Particularly, this means a robust input signal needs to be constructed without relying on extensive prior knowledge of the system.
Such a signal should cover a wide frequency band to excite all dynamics of the system during the experiment.
However, the signal should ensure that systems in different frequency bands can be identified with a specified level of accuracy.

\paragraph{Can we process the input/output data in a non-parametric way that offers more insight than a standard frequency response function?}


\paragraph{Can we }


\section{Outline and Publications}
   The lion's share of this thesis has been published in either peer-reviewed  journals or conferences.
   This section links my different publications to the different sections in this thesis.
   For an overview of publications grouped by type, please refer to page~\pageref{publicationList}.

\begin{refsection}
% http://tex.stackexchange.com/questions/38580/displaying-selected-bibliographic-items-in-the-body-of-the-text
% http://tex.stackexchange.com/questions/126226/how-do-i-instruct-fullcite-to-use-maxbibnames-rather-than-maxcitenames

\makeatletter
\DeclareCiteCommand{\fullcite}
  {\defcounter{maxnames}{\blx@maxbibnames}%
    \usebibmacro{prenote}}
  {\usedriver
     {\DeclareNameAlias{sortname}{default}}
     {\thefield{entrytype}}}
  {\multicitedelim}
  {\usebibmacro{postnote}}
\DeclareCiteCommand{\footfullcite}[\mkbibfootnote]
  {\defcounter{maxnames}{\blx@maxbibnames}%
    \usebibmacro{prenote}}
  {\usedriver
     {\DeclareNameAlias{sortname}{default}}
     {\thefield{entrytype}}}
  {\multicitedelim}
  {\usebibmacro{postnote}}
\makeatother

% http://tex.stackexchange.com/questions/18664/underline-my-name-in-the-bibliography
% Plus see release notes of BIBLATEX 3.3/3.4
\DeclareNameFormat{author}{%
  \nameparts{#1}%
\ifthenelse{\equal{\namepartfamily}{Geerardyn}}%
    {\textbf{\ifblank{\namepartgiven}{}{\namepartgiven\space}\namepartfamily}}%
    {\ifblank{\namepartgiven}{}{\namepartgiven\space}\ifblank{\namepartprefix}{}{\namepartprefix\space}\namepartfamily}%
\ifthenelse{\value{listcount}<\value{liststop}}%
    {\addcomma\space}
    {}}


The quasi-logarithmic multisines presented in \chapref{sec:excitation} are based on a journal article published in \gls{IEEE} Transactions on Instrumentation \& Measurement:
\begin{itemize}
  \item \fullcite{Geerardyn2013TIM}, 
\end{itemize}
preliminary results were also presented at the 2012 \gls{IFAC} symposium on System Identification (\textsc{SYSID}) and the \gls{IEEE} International Instrumentation and Measurement Conference (\textsc{I$^{\text{2}}$MTC}):
\begin{itemize}
  \item \fullcite{Geerardyn2012IMTC}, and
  \item \fullcite{Larsson2012SYSID}.
\end{itemize}
This work has also been presented at the following local (non-refereed) conferences:
\begin{itemize}
  \item \fullcite{Geerardyn2012Benelux}, and
  \item \fullcite{Geerardyn2012ERNSI}.
\end{itemize}

The non-parametric estimation methods in \chapref{sec:nonparametric} are based on a yet-unpublished manuscript.
 The so-called time-truncated \gls{LPM} that is presented in the same chapter is based on a journal published in \gls{IEEE} Transactions on Instrumentation \& Measurement:
\begin{itemize}
  \item \fullcite{Lumori2014TIM}.
\end{itemize}
In particular, this smoother enables one to reduce the effect of noise on an estimated \gls{FRF} using an automated approach.
Relatedly, a preliminary study of the \gls{LPM} in the context of lightly-damped \gls{MIMO} systems has been presented at the (non-refereed) Benelux Meeting on Systems and Control:
\begin{itemize}
    \item \fullcite{Verbeke2015Benelux}.
\end{itemize}

The use of the non-parametric \gls{FRF} estimation methods for \Hinf gain estimation (and in general the \gls{FRF} interpolation from \chapref{sec:hinf}) has been presented at the 2014 \gls{IFAC} World Conference in South Africa and the 2014 Leuven Conference on Noise and Vibration Engineering (ISMA) in Leuven:
\begin{itemize}
  \item \fullcite{Geerardyn2014IFAC},
  \item \fullcite{Geerardyn2014ISMA}.
\end{itemize}
Preliminary results have been presented at local (non-refereed) conferences:
\begin{itemize}
  \item \fullcite{Geerardyn2013Benelux},
  \item \fullcite{Geerardyn2013ERNSI},
  \item \fullcite{Geerardyn2014Benelux},
  \item \fullcite{Geerardyn2014DYSCO}, and
  \item \fullcite{Geerardyn2014ERNSI}.
\end{itemize}
Experimental work related to \chapref{sec:hinf} has also been presented at the 2015 \gls{IFAC} Symposium on System Identification (\textsc{sysid}) and :
\begin{itemize}
  \item \fullcite{Voorhoeve2015SYSID}
  \item \fullcite{Voorhoeve2015ERNSI}
\end{itemize}

The study of different initialization strategies as depicted in \chapref{sec:initvals} has been published in the \gls{IEEE} Transactions on Instrumentation \& Measurement:
\begin{itemize}
  \item \fullcite{Geerardyn2015TIM},
\end{itemize}
and also at the local (non-refereed) 2015 Benelux Meeting on Systems and Control:
\begin{itemize}
    \item \fullcite{Geerardyn2015Benelux}.
\end{itemize}

In collaboration with fellow researchers, I have written a few other publications.
However, these publications are not covered in this dissertation.

\begin{itemize}
  \item \fullcite{vanBerkel2014Automatica},
  \item \fullcite{Cooman2012SMACD},
  \item \fullcite{Cooman2012DYSCO},
  \item \fullcite{Cooman2012ERNSI}.
\end{itemize}
\end{refsection}

