\chapter*{Acknowledgments}
\myEpigraph{We're all going to die, all of us, what a circus! That alone should make us love each other but it doesn't. \\
We are terrorized and flattened by trivialities, we are eaten up by nothing.}{Charles Bukowski}{The Captain is Out to Lunch and \\ the Sailors have taken over the Ship}
% \myEpigraph{In the End, we will remember not the words of our enemies, but the silence of our friends.}{Martin Luther King Jr.}{}

While the preparation of a PhD is for a large part an endeavor of a single person, it is by no means a solitary experience.
Research without collaboration is unthinkable and is unlikely to yield new scientific and technical insights since personal interactions allow new ideas and applications to germinate.

In the first place, I would like to thank my colleagues at the ELEC department for being an outgoing group that has been combining a fond interest for engineering, science, and leisure.
In particular, I am grateful for the great atmosphere in the office and for my office mates over the last few years: Adam, Anna, David, Diane, Ebrahim, Erlyiang, Francesco, Gustavo, Hannes, Maarten, Mark, Matthias, and Maulik (though not all at the same time).
I have fond memories of the time Alexandra and I spend together with Anna and Maarten on our road trip in South Africa and watching the local wildlife.
I have also enjoyed the no-frills company of Adam, his overall good taste as an engineer and dry sense of humor (`nobody expects the Cauchy distribution' was a sparkle of laughter after barren days).
On the diametrical opposite side of the spectrum, but not less agreeable, were my interactions with Matthias.
I really appreciate his deep knowledge and his frankness, which often came into play when the both of us were having rougher times.

Next to research, quite a considerable amount of my time was spent on teaching assignments.
In particular, the many hours we toiled in the `mines of Moria' to aid bachelor students to control pingpong balls went like a breeze thanks to the great company of able-minded colleagues such as Matthias, Evi, Dave, Hans, Sven and Jens.
The great technical and administrative support of Gerd, Johan and Luc have also certainly helped in creating a productive setting for those labs.
While exhausting, I have immensely enjoyed and benefited from those labs that force you to revise basic concepts that I have come to take for granted over the years.

For the circuits and filters course, I have been really lucky to have had a great predecessor in John whom was always available to help out during the first year when I took over the course project together with Laurent in the first year of my PhD.
I have really enjoyed this collaboration since both of us brought very complementary skills to the project.
I am also very grateful for Alexander and Dries for being great colleagues and taking over a larger chunk of this workload when I started writing my dissertation.

%FIXME proffen, technici, secretaressen

%FIXME mikaya, John
%FIXME istvan

%FIXME Johan
%FIXME Tom

%FIXME Matthijs
%FIXME CST
%FIXME Rick, Annemiek, Frank,  MST

I would also like to express my sincere gratitude for my jury member for their flexibility and great effort in aiding me to ameliorate this dissertation and my understanding of connections of my work to other fields of science and engineering.
In retrospect, I have absolutely enjoyed discussing with them during my private defense and the few hours afterwards.
In particular, I have been astounded by the huge amount of work that Koen put into being the best secretary that I would not even have dared dreaming of.
Also, I appreciate Tomas for taking the time to go through the draft version of this dissertation with me in painstaking detail in the hours after my private defense.

This book also would not have been possible without my fellow members of the matlab2tikz team: Nico, Oleg, Peter, Michael and a handful of incidental other helpers.
Most of the graphs in this text have, at one point or another, passed through the matlab2tikz codebase.
I have been really enjoying our collaboration, getting-things-done attitude, and discussions regarding good coding practices in \MATLAB{} and the future directions of the matlab2tikz project.

%FIXME family: oma, opa, oma spanje, mama, papa, lara, youri
%FIXME family-in-law
%FIXME alexandra

And last, but not least, I would like to thank our cat Pixie for being an enjoyably crazy critter and for his sounding enthusiasm when bringing home \emph{little presents} and leaving them either on the floor or on my pillow while I'm fast asleep.
His intent has been duly noted and appreciated (the nightly hours and contents of those presents, however, were not always equally well-received).
