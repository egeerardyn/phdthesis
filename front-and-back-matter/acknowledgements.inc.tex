\chapter*{Acknowledgments}
\vspace{-1em}
\myEpigraph{We're all going to die, all of us, what a circus! That alone should make us love each other but it doesn't. \\
We are terrorized and flattened by trivialities, we are eaten up by nothing.}{Charles Bukowski}{The Captain is Out to Lunch and \\ the Sailors have taken over the Ship}
\vspace{-1.5em}

% \small{
{
\small
\sffamily

While a PhD only has a single author, it is by no means a solitary experience.
Research without collaboration is unlikely to yield new scientific and technical insights since personal interactions allow new ideas and applications to germinate.

In the first place, I would like to thank my colleagues at the ELEC department for being an outgoing group that has been combining a fond interest for engineering, science, and leisure.
In particular, I am grateful for the great atmosphere in the office and for my office mates over the last few years: \textsc{Adam}, Anna, David, Diane, Ebrahim, Erlyiang, Francesco, Gustavo, Hannes, Maarten, Mark, Matthias, and Maulik (though not all at the same time).
I have fond memories of the time Alexandra and I spend together with Anna and Maarten on our road trip in South Africa and going on a safari to see the local wildlife and smell the wild lavender.
I have also enjoyed the no-frills company of \textsc{Adam}, his overall good taste as an engineer and dry sense of humor (`nobody expects the Cauchy distribution' was a sparkle of laughter after barren days).
I really appreciate Matthias's deep knowledge and his frankness, especially when the both of us were having rougher times.
Also, the interaction I had with Cedric, often concerning good microcontroller code, have always enlightened my day.
Apart from the actual research, I had a very good time on social activities with Péter who guided us through Vienna, David who was one of the main organizers of many social activities, and many of the other people at ELEC.

Next to research, quite a considerable amount of my time was spent on teaching assignments.
In particular, the many hours we toiled in the `mines of Moria' to aid bachelor students to control pingpong balls went like a breeze thanks to the great company of able-minded colleagues such as Matthias, Evi, Dave, Hans, Sven and Jens.
The great technical and administrative support of Gerd, Johan and Luc have also certainly helped in creating a productive setting for those labs.
While exhausting, I have immensely enjoyed and benefited from those labs that force you to revise basic concepts that I have come to take for granted over the years.

For the circuits and filters course, I have been really lucky to have had a great predecessor in John whom was always available to help out during the first year when I took over the course project together with Laurent in the first year of my PhD.
I have really enjoyed this collaboration since both of us brought very complementary skills to the project.
I am also very grateful for Alexander and Dries for being great colleagues and taking over a larger chunk of this workload when I started writing my dissertation.

During the few summers that I spent at ELEC, I have been fortunate enough to have collaborated with Mikaya and John, on what has become a considerable part of this dissertation.
I really admire Mikaya's \latin{joie de vivre}, which really made his visits to Brussels a time to look forward to each year.
Also, I really love the well-balanced way in which John oversaw this work; I am convinced that in academia, more people like John are needed.

In general, I would like to thank the professors of ELEC: Rik, Yves, Gerd, Ivan, and Leo.
Over the last years, I have come to learn Rik as one of the most intelligent, correct and friendly people of ELEC.
Gerd is a scientific jack-of-all trades with whom I share a strong love for beer, barbecues and finding practical solutions in the lab.
And Yves is one of the people without whom I would not have even started working towards a PhD.
I have dearly enjoyed our discussions on software design and philosophy, your very critical --yet constructive-- view on some of my papers, and your fascination in the color of my beverages during lunch.
Too often, we tend to forget that people in supporting roles are invisible most of the time.
So I would like to take this space to also thank our secretaries (Ann, Bea and Annick), our technician Johan and system administrator Sven for running a tight ship most of the time.
I also would like to thank Leo, for his political courage and taking on responsibilities when the times call for it.
I genuinely appreciate his efforts to provide a stable environment in the department such that research can be done.

During my first real international conference, I was happy to meet with István Kollár, who has unfortunately passed away recently. 
I have really enjoyed our scientific discussions, grabbing some beers, and the view he offered into the foregone era of the Iron Curtain.

During these last few years, I have been working under the close supervision of my promotor Johan.
Although our personalities differ wildly, I truly appreciate Johan for his often refreshing views on the technical content, for his broad overview of engineering and the interesting discussion we have had over the years.
I am really grateful, since many of the encounters I had over these last few years, were either enabled and/or strongly encouraged by him.

In the second year of my PhD, I have been a visiting researcher in the Control Systems Theory group of prof. Maarten Steinbuch.
I remember my first encounter with Maarten as a friendly, but rather critical, meeting which took me aback at first.
I have great admiration for Maarten's very enjoyable mix of scientific accuracy, critical business acumen and lightheartedness.
During these months at the CST group, I much enjoyed the company of Matthijs, Frank, Annemiek and Rick.
I was very happy to have had the chance to interact with Rick who had an excellent understanding of the set-up that I was working on and who was always available to discuss or joke around.
Also, I took great pleasure in interacting with Matthijs: during his time at ELEC, my time at CST, and all of the times we hung out as colleagues or as friends.
Overall, I would like to thank all the people (PhD students, professors, secretaries, ...) at CST for their open and warm atmosphere.

However, my main scientific interactions in Eindhoven have been with Tom, who eventually became my second promotor.
I have always enjoyed our discussions and the broad and deep scientific knowledge he masters and yet is able to explain with clarity.
Thanks to his very open attitude, I was able to learn a whole lot in quite a short time.

I would also like to express my sincere gratitude towards my jury member for their flexibility and great effort in aiding me to ameliorate this dissertation, and my understanding of connections of my work to other fields of science and engineering.
In retrospect, I have absolutely enjoyed discussing with them during my private defense and the few hours afterwards.
In particular, I have been astounded by the huge amount of work that Koen put into being a great secretary.
Also, I appreciate Tomas for taking the time to go through the draft version of this dissertation with me in painstaking detail in the hours after my private defense.

This book also would not have been possible without my fellow members of the matlab2tikz team: Nico, Oleg, Peter, Michael and a handful of incidental other helpers.
Most of the graphs in this booklet have passed through the matlab2tikz codebase.
I have been really enjoying our collaboration, getting-things-done attitude, and discussions regarding good practices for making figures and coding in \MATLAB{}, and the future directions of the project.

During my time as a bachelor and master student, I have always enjoyed working together with Jeroen.
During both our PhD's at different locations, I have greatly valued how we kept in touch and tried to boost each other's spirits during the darker days of our academic activities and sometimes personal struggles.

% \newpage
At home, I would like to thank my family: my mom and dad for providing a stable yet warm environment, even though not everything has been clear sailing.
I really appreciate how the both of you have always found a way to make the best out of it and how I could always count on you.
Also my sister Lara and her husband Youri, for being nice people on whom you can always count.
Although we are completely different in our professions and thinking, I place great value in getting along very well, despite those differences.
I would also like to thank my grandparents (Jo and Ria) for on the one hand bringing me into contact with mathematics from a young age, and for their supportiveness.
Also, I really appreciate the great lengths my other grandmother goes through to physically unite the greater Geerardyn family on numerous occasions.

Next to my ``original'' family, I am very grateful to my family `in-law' (despite not being married): Linda and Dirk and their kin.
As with my own parents, I was able to count on the both of you when times got tough (or when barbecues were involved).

The last person I would like to thank is my girlfriend Alexandra.
Without her encouragement, I am convinced I would never have finished my PhD.
I am grateful that you pushed me to not quit and that you did everything possible to bring this chapter in our lives to a good end, even though you yourself have not been void of concerns about your PhD.

And last, but not least, I would like to thank our cat Pixie for being an enjoyably crazy critter and for his sounding enthusiasm when bringing home \emph{little presents} and leaving them either on the floor or on my pillow while I'm fast asleep.
His intent has been duly noted and appreciated (the nightly hours and contents of those presents, however, were not always equally well-received).
}
