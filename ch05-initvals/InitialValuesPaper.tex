% % BEGIN OF TIKZ ONLY
% \usepackage{tikz,pgfplots}
% \usepackage{pgfplotstable}
% \pgfplotsset{compat=newest}
% \usetikzlibrary{external}
% %\tikzset{external/optimize=false}
% \tikzset{external/mode=list and make}
% \tikzexternalize[prefix=tikz/]
% \tikzsetfigurename{figure}
% \definecolor{LRM}{named}{TangoScarletRed3}
\definecolor{LPM}{named}{TangoSkyBlue2}
\definecolor{SA}{named}{TangoChameleon2}
\definecolor{truesys}{named}{TangoAluminium4}
\definecolor{validation}{named}{TangoChameleon2}

\definecolor{LPMshade}{named}{TangoSkyBlue1}
\definecolor{LRMshade}{named}{TangoScarletRed1}
\definecolor{refColor}{named}{TangoAluminium6}

\definecolor{frf}{named}{TangoButter3}
\definecolor{param}{named}{TangoScarletRed2}

\pgfplotsset{egon/.style={line join=round}}

\pgfplotsset{truesys/.style={egon,color=truesys,densely dotted,line width=1.5pt}}

\pgfplotsset{hinfnorm/.style={egon,dashed}}
\pgfplotsset{interpol/.style={egon,sharp plot,solid,mark options={solid}}}

\pgfplotsset{LRMgrid/.style={color=LRM,mark size=1.5pt,only marks,mark=square*,mark options={solid}}}
\pgfplotsset{LPMgrid/.style={color=LPM,mark size=1.5pt,only marks,mark=*,mark options={solid}}}
\pgfplotsset{SAgrid/.style={color=SA,densely dotted,solid,mark=triangle*,mark size=1.5pt,mark options={solid}}}

\pgfplotsset{smallmarkers/.append style={mark size=0.5pt}} %TODO: replace smallmarkers -> medsmallmarkers

\pgfplotsset{frf/.style={line join=round,color=frf,solid,mark options={solid},mark size=0.2pt}} %TODO: replace

\pgfplotsset{paramPhat/.style={egon,color=param,solid,line width=1.2pt}}


\pgfplotsset{validation/.style={color=validation,densely dotted,mark=diamond*,mark size=1pt,mark options={solid}}}

\pgfplotsset{shaded/.style={opacity=0.5,area legend,solid}}

\pgfplotsset{LPMshade/.style={shaded,fill=LPMshade,draw=LPM}}
\pgfplotsset{LRMshade/.style={shaded,fill=LRMshade,draw=LRM}}

\pgfplotsset{reference/.append style={color=refColor,densely dotted,line width=0.75pt}}
\pgfplotsset{gammafrf/.style={color=black,mark=pentagon,mark options={solid}}}

% % END OF TIKZ ONLY
% \newcommand{\legref}[2][1]{(\ref{#2})}

%  \newlength\figureheight
%  \newlength\figurewidth
%  \setlength\figureheight{7cm}
%  \setlength\figurewidth{10cm}

%  \newlength\onecolumnwidth
%\title{Improved Initial Estimates Via FRF Smoothing Techniques for Parametric Identification of LTI Systems}

% \begin{abstract}
% Good initial values are crucial to obtain solutions of non-convex optimization problems. When estimating the transfer function of physical systems from measured noisy data, obtaining good initial parameter estimates is therefore a primordial step.
% In this paper, it is shown that smoothing the measured frequency response function (FRF) of a linear time-invariant system enhances the construction of initial estimates significantly, resulting in the optimization schemes to converge to a better optimum. 
% This is achieved with minimal user interaction.

% Two smoothing techniques, the time-truncated local polynomial method (LPM) and the regularized finite impulse response (RFIR), are compared with the existing generalized total least squares (GTLS) and the bootstrapped total least squares (BTLS) initial estimates. 
% The improvement attributable to smoothing is demonstrated by a simulation and by measurements of an electrical filter. 
% The results ultimately show that the parametric models obtained using the proposed starting values are much more likely to give a good description of the measured system and hence lead to more useful models.
% \end{abstract}

\setlength\onecolumnwidth{\columnwidth}

%\comment{JL: 
%Focus of the article (this is what the article must reflect):

%\begin{itemize}
%\item demonstrate that a smoothed FRF, obtained as described in \citep{Lumori2014}, used as an initial estimate increases the success rate of a parametric LTI TF identification
%\item
%\comment{JL:  emphasize the importance of extending the SNR range for which the success rate is acceptable, yielding a reliable parametric estimator to be used with as little user interaction as possible.
%\end{itemize}
%}
