
\newcommand{\reviewer}[1]{\paragraph{#1}}

\reviewer{Reviewer \#2 (Remarks for the Author):}
\begin{quote}
The authors propose two smoothing techniques, which improve the initial estimates for transfer function identification. Figure 2 and section VI are appreciated. 

Comments: 

1. Please treat carefully the optimization problems. Just presenting the cost functions is not enough. 

3. Which are the optimization algorithms? Clear steps must be provided. 
\end{quote}
Note that the minimization of the cost function (5) in the paper has been extensively studied in the literature (e.g. \cite[Section 9.11]{Pintelon2001} which has been referenced in the manuscript).
We have used the well-known Levenberg-Marquardt optimization algorithm to perform the non-convex optimization.
This information has also been added to the paper.

\begin{quote}
2. Do your smoothing techniques guarantee and/or keep the minimum? 
\end{quote}
The cost function involving the smoothed estimate is different from the cost function with the raw data. 
Obviously, in that case the minimum is expected to be shifted.
However, the result from this cost function is used as an \emph{initial} estimate for the cost function with the raw data. 
In the end, only the cost function with the raw data is retained.
The initial estimate does \emph{not} change the minimum of a cost function.

\begin{quote}
4. The transparency of the paper is suffering because the authors do not present how the theoretical results from sections II and III are applied in section IV. 

5. The same comment is also applied for section V, i.e., low transparency. 
\end{quote}

The theoretical considerations in sections II and III are applied in section IV.
In section III, the initial estimation procedures are enumerated, and a distinct notation for each of them is given:
\begin{itemize}
  \item use of the true parameters: $\model{0}$,
  \item use of existing techniques: $\model{exist}$,
  \item use of the LPM with truncation: $\model{trunc}$,
  \item use of the RFIR: $\model{RFIR}$.
\end{itemize}
Then, the results of the latter three initialization techniques are compared to the results with $\model{0}$ in Section IV (simulations) and Section V (measurements).

\begin{quote}
6. I am not sure if section IV is actually necessary. You may drop it out. 
\end{quote}

The presence of section IV is crucial for the article, and will not be skipped, for the following reason. 
The goal of the article is to demonstrate that the use of smoothing techniques increases the success rate the minimization of the ML cost function. 
While Sections II and III introduce the methods used, Section IV actually \emph{does} the demonstration of the improvement. 
It is important to include the simulation results, because these allow to make Monte Carlo simulations at various SNR values, to obtain statistically meaningful results. 
It would be much more difficult to apply a Monte Carlo method to the measurements, since a huge number of experiments must be performed, with different experimental conditions (SNR), on a variety of different systems.

\begin{quote}
2) It is unclear how generalizable the procedure introduced by Authors is, and what is its domain of applicability. Based on section IV Demonstration (simulated data), a reader may be convinced that RFIR is the absolute winner.

So, it comes as an unpleasant surprise in section V Experimental Results to learn that RFIR only wins in $50\%$ of cases (Table I). The natural question of ``why so different'' is never discussed. It seems that a lack of ground truth in a real experiment is correctly addressed by using a response from an almost noiseless system (60dB, p.18). Could it be that in simulation two low-pass filters were investigated (each with $n_{\theta} = 5$ parameters, p.13), while in experiment pass-band filter was tested with $n_{\theta} = 9$ adjustable parameters? Having the same system tested in simulations and in experiment could partially answer this question. 

However, the results in Table I do not necessarily nullify a usefulness of the RFIR as a starting point. It may be that in the cases when $\model{RFIR}$ is not the best estimate, a difference between $\model{best}$ and $\model{RFIR}$ is acceptably small. If for cases when $\model{best} = \model{RFIR}$ is much better than the second best estimate $G_{2nd}$, then it makes sense to always use RFIR, as a penalty for wrong choice would be mild. 

\end{quote}

The fact that ‘only’ $50\%$ of the RFIR estimates turn out to be the best estimates does not mean that the remaining $50\%$ from RFIR are bad estimates. 
It simply means that, for the remaining $50\%$, the other methods (trunc and exist) happened to give a slightly smaller cost function. 
Since Table I is misleading, we have decided to remove it from the paper.

In \figref{fig:overview} we can compare $\model{2nd}$ and $\model{best}$.
By inspecting the different estimates for  each of the repeated experiments (\figref{fig:overview}), it can easily be seen that choosing $\model{RFIR}$ while it is not $\model{best}$ only leads to a very modest performance degradation.
In \figref{fig:histogram}, the different methods are compared in this respect by subtracting the performance of the $\model{best}$ from the performance of the respective methods.
From that figure it can be seen that there is little performance to gain \emph{or} lose from choosing $\model{RFIR}$ over $\model{best}$ when RFIR is not the `best' method (as measured by the cost function $V$).
The outcome is a lot less favorable for the alternative methods.

\begin{figure}
  \centering
  \setlength{\figurewidth}{0.85\columnwidth}
  \setlength{\figureheight}{0.68\figurewidth}
  % This file was created by matlab2tikz.
%
%workdir  : /Users/egon/Dropbox/VUB/PhD/Mikaya/initialvalues
%stack    : ../../../../../../../../Users/egon/Dropbox/VUB/PhD/Mikaya/initialvalues/processAllMeasurements (10)
%git info : commit 45c2d89a4e4044db9de2834c5c1c86bec69e4864
%           Author: Egon Geerardyn <egon.geerardyn@gmail.com>
%           Date:   Tue Oct 7 09:18:45 2014 +0200
%           
%               estimParamMeas: duplicate code removed
%           
%            estimateParametricMeas.m | 9 +--------
%            1 file changed, 1 insertion(+), 8 deletions(-)
%           
%
%
\begin{tikzpicture}
\begin{axis}[%
name=fit,
width=\figurewidth,
height=0.5\figureheight,
scale only axis,
xmin=1,
xmax=100,
xmajorgrids,
xtick={15,23,41,53,69,88,100},
xticklabels={},
ymode=log,
ymin=4000,
ymax=100000,
yminorticks=true,
ylabel={Cost $V(\model{\bullet})$},
ymajorgrids,
yminorgrids,
axis x line*=bottom,
axis y line*=left,
]
\addplot [best,line width=1pt] table[]{\thisDir/data/overview-cost-val/cost-best.tsv};
\label{leg:best}
% \addlegendentry{$\hat{G}_{best}$}

\addplot [RFIR,RFIRmark,only marks] table[]{\thisDir/data/overview-cost-val/cost-rfir.tsv};
\label{leg:RFIR}
% \addlegendentry{$\hat{G}_{RFIR}$}

\addplot [LPMTrunc,LPMTruncInitmark,only marks] table[]{\thisDir/data/overview-cost-val/cost-trunc.tsv};
\label{leg:trunc}
% \addlegendentry{$\hat{G}_{trunc}$}

\addplot [existing,existingInitmark,only marks] table[]{\thisDir/data/overview-cost-val/cost-exist.tsv};
\label{leg:exist}
% \addlegendentry{$\hat{G}_{existing}$}

\addplot [G0Hat,G0Hatmark,only marks,medsmallmarkers] table[]{\thisDir/data/overview-cost-val/cost-true.tsv};
% \addlegendentry{$\hat{G}_0$}
\label{leg:VXI}

\end{axis}


\begin{axis}[%
name=valid,
width=\figurewidth,
height=0.5\figureheight,
anchor=above north west,
at={(fit.below south west)},
scale only axis,
xmin=1,
xmax=100,
xmajorgrids,
xtick={15,23,41,53,69,88,100},
ymode=log,
ymin=0.4,
ymax=90,
yminorticks=true,
xlabel={Repetition},
ylabel={$\norm[2]{\model{\bullet} - \model{VXI}}$},
ymajorgrids,
yminorgrids,
axis x line*=bottom,
axis y line*=left
]
\addplot [dashed] table[]{\thisDir/data/overview-cost-val/sys-h2norm.tsv};
\label{leg:reference}
\addplot [best,line width=1pt] table[]{\thisDir/data/overview-cost-val/distance-best.tsv};
\addplot [RFIR,RFIRmark,only marks] table[]{\thisDir/data/overview-cost-val/distance-rfir.tsv};
\addplot [LPMTrunc,LPMTruncInitmark,only marks] table[]{\thisDir/data/overview-cost-val/distance-trunc.tsv};
\addplot [existing,existingInitmark,only marks] table[]{\thisDir/data/overview-cost-val/distance-exist.tsv};
\addplot [G0Hat,G0Hatmark,only marks,medsmallmarkers] table[]{\thisDir/data/overview-cost-val/distance-true.tsv};
\end{axis}
\end{tikzpicture}%

  \caption{Cost function $V(\model{\bullet})$ and validation distance of the model estimated in each repetition of the measurement.
  The same data as in Fig.~10 and Fig.~11 of the paper is presented.
  Based on the cost function, $\model{RFIR}$~\legref{leg:RFIR} is not always the best estimate $\model{best}$~\legref{leg:best}.
  Especially in the validation plot (bottom), $\model{RFIR}$ performs (almost) as well or even better than $\model{exist}$~\legref{leg:exist} and $\model{trunc}$~\legref{leg:trunc}.
  Both latter methods often produce models that perform poorly compared to $\| G_{VXI} \|$~\legref{leg:reference} and $\model{VXI}$~\legref{leg:VXI}.
  This means that in our limited experimental study, the performance degradation of choosing RFIR to produce initial values over the studied alternatives is negligible.}
  \label{fig:overview}
\end{figure}

\begin{figure}
  \centering
  \setlength{\figurewidth}{0.75\columnwidth}
  \setlength{\figureheight}{0.60\figurewidth}
  % This file was created by matlab2tikz.
%
%workdir  : /Users/egon/Dropbox/VUB/PhD/Mikaya/initialvalues
%stack    : ../../../../../../../../Users/egon/Dropbox/VUB/PhD/Mikaya/initialvalues/processAllMeasurements (10)
%git info : commit 45c2d89a4e4044db9de2834c5c1c86bec69e4864
%           Author: Egon Geerardyn <egon.geerardyn@gmail.com>
%           Date:   Tue Oct 7 09:18:45 2014 +0200
%
%               estimParamMeas: duplicate code removed
%
%            estimateParametricMeas.m | 9 +--------
%            1 file changed, 1 insertion(+), 8 deletions(-)
%
%
%
\begin{tikzpicture}
\pgfplotsset{every axis title/.append style={font={\footnotesize},at={(rel axis cs:0.5,0.97),anchor=center}}}
\begin{axis}[%
name=Estim,
width=0.5\figurewidth,
height=\figureheight,
scale only axis,
unbounded coords=jump,
xmin=0.5,
xmax=3.5,
grid=both,
title={Estimation: $V_{\bullet} = V(\model{\bullet})$ $\vphantom{\norm[2]{\model{\bullet}}}$},
x tick label as interval,
xtick={0.5,1.5,2.5,3.5},
xticklabels={{$\model{exist}$},{$\model{trunc}$},{$\model{RFIR}$}},
ymode=log,
ymin=0.001,
ymax=100000,
yminorticks=true,
ylabel={$V_{\bullet} - V_{\mathrm{best}}$},
axis background/.style={fill=white},
axis x line*=bottom,
axis y line*=left
]

\addplot[degradationBox, forget plot] table[] {\thisDir/data/special-histogram/estim-exist-degradation-box.tsv}--cycle;
\addplot [degradationMedian, forget plot] table[]{\thisDir/data/special-histogram/estim-exist-degradation-median.tsv};
\addplot [degradationMarks, forget plot] table[]{\thisDir/data/special-histogram/estim-exist-degradation-data.tsv};
\addplot[degradationBox, forget plot] table[] {\thisDir/data/special-histogram/estim-trunc-degradation-box.tsv}--cycle;
\addplot [degradationMedian, forget plot] table[]{\thisDir/data/special-histogram/estim-trunc-degradation-median.tsv};
\addplot [degradationMarks, forget plot] table[]{\thisDir/data/special-histogram/estim-trunc-degradation-data.tsv};

\addplot[degradationBox, forget plot] table[] {\thisDir/data/special-histogram/estim-RFIR-degradation-box.tsv}--cycle;
\addplot [degradationMedian, forget plot] table[]{\thisDir/data/special-histogram/estim-RFIR-degradation-median.tsv};
\addplot [degradationMarks, forget plot] table[]{\thisDir/data/special-histogram/estim-RFIR-degradation-data.tsv};

\addplot [percentileLines, forget plot] table[]{\thisDir/data/special-histogram/estim-max.tsv};
\addplot [percentileLines, forget plot] table[]{\thisDir/data/special-histogram/estim-median.tsv};
\addplot [percentileLines, forget plot] table[]{\thisDir/data/special-histogram/estim-min.tsv};
\end{axis}

\begin{axis}[%
name=Valid,
anchor=north west,
at={($(Estim.north east) + (1em,0pt)$)},
width=0.5\figurewidth,
height=\figureheight,
scale only axis,
unbounded coords=jump,
xmin=0.5,
xmax=3.5,
grid=both,
x tick label as interval,
xtick={0.5,1.5,2.5,3.5},
xticklabels={{$\model{exist}$},{$\model{trunc}$},{$\model{RFIR}$}},
ymode=log,
ymin=0.0001,
ymax=100,
yminorticks=true,
ylabel={${D_{\bullet} - D}_{\mathrm{best}}$},
title={Validation: $D_{\bullet} = \norm[2]{\model{\bullet} - G_{\mathrm{VXI}}}$},
axis x line*=bottom,
axis y line*=right,
% legend to name=leg:init:secondBest,
legend columns=-1,
legend cell align=center,
legend style={draw=none, at={(rel axis cs:-0.05,1.1)},anchor=south}
]

\addplot[enhancementBox, forget plot] table[] {\thisDir/data/special-histogram/valid-exist-enhancement-box.tsv}--cycle;
\addplot [enhancementMedian, forget plot] table[]{\thisDir/data/special-histogram/valid-exist-enhancement-median.tsv};
\addplot [enhancementMarks, forget plot] table[]{\thisDir/data/special-histogram/valid-exist-enhancement-data.tsv};

\addplot[enhancementBox, forget plot] table[] {\thisDir/data/special-histogram/valid-trunc-enhancement-box.tsv}--cycle;
\addplot [enhancementMedian, forget plot] table[]{\thisDir/data/special-histogram/valid-trunc-enhancement-median.tsv};
\addplot [enhancementMarks, forget plot] table[]{\thisDir/data/special-histogram/valid-trunc-enhancement-data.tsv};

\addplot[enhancementBox] table[] {\thisDir/data/special-histogram/valid-RFIR-enhancement-box.tsv}--cycle;
%\addlegendentry{boxplot $X_{\bullet} < X_{\mathrm{best}}$}
\addlegendentry{and}

\addplot [enhancementMedian, forget plot] table[]{\thisDir/data/special-histogram/valid-RFIR-enhancement-median.tsv};
% \addlegendentry{$X_{\bullet} < X_{\mathrm{best}}$}

\addplot [enhancementMarks] table[]{\thisDir/data/special-histogram/valid-RFIR-enhancement-data.tsv};
\addlegendentry{$X_{\bullet} < X_{\mathrm{best}}$}

\addlegendimage{empty legend}
\addlegendentry{\vspace{2em}}

\addplot[degradationBox, forget plot] table[] {\thisDir/data/special-histogram/valid-exist-degradation-box.tsv}--cycle;
\addplot [degradationMedian, forget plot] table[]{\thisDir/data/special-histogram/valid-exist-degradation-median.tsv};
\addplot [degradationMarks, forget plot] table[]{\thisDir/data/special-histogram/valid-exist-degradation-data.tsv};

\addplot[degradationBox, forget plot] table[]{\thisDir/data/special-histogram/valid-trunc-degradation-box.tsv}--cycle;
\addplot [degradationMedian, forget plot] table[]{\thisDir/data/special-histogram/valid-trunc-degradation-median.tsv};
\addplot [degradationMarks, forget plot] table[]{\thisDir/data/special-histogram/valid-trunc-degradation-data.tsv};

\addplot[degradationBox] table[] {\thisDir/data/special-histogram/valid-RFIR-degradation-box.tsv}--cycle;
% \addlegendentry{boxplot $X_{\bullet} > X_{\mathrm{best}}$}
\addlegendentry{and}

\addplot [degradationMedian, forget plot]  table[]{\thisDir/data/special-histogram/valid-RFIR-degradation-median.tsv};
% \addlegendentry{$X_{\bullet} > X_{\mathrm{best}}$}

\addplot [degradationMarks]  table[]{\thisDir/data/special-histogram/valid-RFIR-degradation-data.tsv};
\addlegendentry{$X_{\bullet} > X_{\mathrm{best}}$}

\addlegendimage{empty legend}
\addlegendentry{\vspace{2em}}

\addplot [percentileLines, forget plot]  table[]{\thisDir/data/special-histogram/valid-max.tsv};
\addplot [percentileLines, forget plot]  table[]{\thisDir/data/special-histogram/valid-median.tsv};
\addplot [percentileLines]   table[]{\thisDir/data/special-histogram/valid-min.tsv};
\addlegendentry{$\min, \mathrm{median}, \max \; X_{\bullet}$}

\end{axis}
\end{tikzpicture}%

  \caption{
  Performance degradation/enhancement ($X_{\bullet}-X_{\mathrm{best}}$) for choosing a given method $\bullet$ instead of the `best' method for the measurement example in the paper.
  In red/orange the positive values ($X_{\bullet} > X_{\mathrm{best}}$), the performance degradation for using $\model{\bullet}$ over $\model{best}$ is shown.
  The negative values ($X_{\bullet} < X_{\mathrm{best}}$), i.e. the performance enhancement (for using $\model{\bullet}$ instead of $\model{best}$) is shown in green/blue.
  Zero values (i.e. $X_{\bullet} = X_{\mathrm{best}}$) are not shown, hence the data shows when method $\bullet$ is \emph{not} selected as the `best' method.
  The dashed lines are the minimum, median and maximum of $X_{\bullet}$ to give a sense of scale.
  The performance degradation of choosing $\model{RFIR}$ instead of $\model{best}$ is smaller than the corresponding performance degradation for the other methods.
  The performance enhancement from choosing $\model{RFIR}$ over $\model{best}$, is almost on par with the corresponding performance degradation (i.e. there is little to gain \emph{or} lose from choosing $\model{RFIR}$ over $\model{best}$).
  For $\model{trunc}$ and $\model{exist}$, there is much less performance to gain from picking those over $\model{best}$ and there is a lot more to lose.
  }
  \label{fig:histogram}
\end{figure}

\begin{quote}

The use of a fixed method would greatly simplify an algorithm design in any standalone autonomous toolbox. The results in Table II do not help to uncover whether such a strategy is permissible. To solve this dilemma, two histograms of differences $\Delta = \model{best} - \model{2nd}$ need to be built and shown: one for $\model{best} = \model{FRIR}$ and the other for $\model{best} \neq \model{FRIR}$. 
\end{quote}

We agree that a single method would greatly simplify the implementation of a toolbox.
However, we think that using multiple initial value strategies is a valuable approach since this makes a toolbox more robust to different conditions (SNR, frequency resolution, system order, \ldots) as it reduces the likelihood that a local optimum is hit.
This can e.g. be seen in subplot 2 of \figref{fig:distancesStress} (which is discussed further on).
A similar approach w.r.t. to combine initial values is followed in some toolboxes, e.g. \cite{FDIDENT,TDIDENT}.

Many nature-inspired optimization algorithms (such as Particle Swarm Optimization (PSO)) and stochastic optimization algorithms even use hundreds or thousands of randomized initial values, in part to avoid local optima.
But, obviously, using many initial values comes at a (considerable) increase in computational cost.

While this discussion and this inherent trade-off is very interesting, we think this is outside of the scope of the paper where we only wish to demonstrate that smoothing techniques such as RFIR are a worthwhile source of initial values, but not necessarily the only one.

\begin{quote}
The abstract promises a better starting point for a minimization of a non-convex cost function. However, no effort is undertaken in the paper to show that improvement is expected ``in most cases'' (or, under which conditions). The boundaries between different domains of starting points leading to different local minima may be very complicated. In fact, the higher the dimension of a search space is ($n_{\theta}$), the more likely it is that the boundaries are more complicated. 

It is obvious that in some cases where subsets of starting points stretch over the boundaries, a particular strategy for selecting a starting point may lead to improvement. However, the paper doesn't explain why this should happen ``in most cases''. A reader would gain more confidence in a robustness of the introduced method if its boundary of applicability were clearly spelled out (for example: the method applicable to filters of order less than ... based on extensive testing). 
\end{quote}

We agree that the complicated boundaries between attraction regions of the different local optima of a cost function makes this a hard problem.
Moreover, the intricate relationship between the actual pole locations, signal SNR, choice of excitation signal, \ldots and shape of the cost function makes that determining clear boundaries on the usability of a starting value seems intractable to us.

On the one hand, an obvious limitation to the presented initialization techniques is that they share the limitations of the smoothing techniques: if the non-parametric estimate is a worse representation than the raw data (e.g. heavily biased), it is very unlikely that the initial estimate will outperform the existing estimates.
In particular for RFIR, a simulation study in~\cite{Chen2013} suggests that RFIR handles systems with model orders up to at least $30$, even for small datasets $N\leq 500$.
For LPM (with or without time-truncation), no comparable studies are available to our knowledge, but the LPM itself has been used successfully in diverse practical applications.

For the parametric estimates, we ran some additional simulations to verify the usability of the proposed initial values.
In particular, we used
\begin{itemize}
  \item low-pass, high-pass, band-pass and band-stop 
  \item Chebyshev Type I, Chebyshev Type II, Elliptical and Butterworth
\end{itemize}
discrete-time filters (see \figref{fig:bodeplots}) of tenth degree generated by the MATLAB functions \texttt{cheby1}, \texttt{cheby2}, \texttt{ellip} and \texttt{butter} and an SNR of $20 \unit{dB}$ and $N=1024$ measured samples.
Note: the band-pass and band-stop filters actually have a model order of 20.

The input excitation and the disturbing output noise were both white and Gaussian.
The results of a Monte Carlo simulation consisting of $100$ runs (as described above) have been summarized qualitatively in the paper (due to limited space) and are discussed below in more detail.

\TODO{add table with different transfer functions, generation code, etc. for stress test}

\begin{figure}[p]
  \setlength{\figurewidth}{0.75\columnwidth}
  \setlength{\figureheight}{0.6\figurewidth}
  \centering
  %\ref{legBodes}
  % This file was created by matlab2tikz.
%
%workdir  : /Users/egon/Dropbox/VUB/PhD/Mikaya/initialvalues
%stack    : ../../../../../../../../Users/egon/Dropbox/VUB/PhD/Mikaya/initialvalues/testMonteCarloSystemsSmall (10)
%git info : commit 7bee89de2a914143a08c391cf0cbb846b1db5777
%           Author: Egon Geerardyn <egon.geerardyn@gmail.com>
%           Date:   Mon Feb 2 18:15:16 2015 +0100
%           
%               lo SNR stress test with Cheby
%           
%            printSummary.m               | 10 +++++----
%            summary.m                    |  2 ++
%            testMonteCarloSystemsSmall.m | 51 ++++++++++++++++++++++----------------------
%            3 files changed, 34 insertions(+), 29 deletions(-)
%           
%
%
%

\tikzset{plotnum/.style={color=black!30,align=left,font=\scriptsize,anchor=north west}}
\pgfplotsset{cheby1/.style={ymin=-80,ymax=10,ytick={0,-60},extra y ticks={10},extra y tick labels={}}}
\pgfplotsset{cheby2/.style={ymin=-60,ymax=5,ytick={0,-40},extra y ticks={5},extra y tick labels={}}}
\pgfplotsset{ellip/.style={ymin=-80,ymax=10,ytick={0,-60},extra y ticks={10},extra y tick labels={}}}
\pgfplotsset{butter/.style={ymin=-80,ymax=15,ytick={0,-60},extra y ticks={15},extra y tick labels={}}}

\begin{tikzpicture}

\pgfplotsset{every axis legend/.style={anchor=west}}

\begin{groupplot}[%
group style={%
  group name=bodes,
  group size=4 by 4,
  horizontal sep=0.2cm,
  vertical sep=0.2cm,
  x descriptions at=edge bottom,
  y descriptions at=edge left},
xlabel={$\omega \axisunit{rad/s}$},
scale only axis,
xmin=0,
xmax=3.1415,
xtick={0,1.2566,1.5708,1.885,3.1416},
xticklabels={{},{$\frac{2\pi}{5}$},{},{$\frac{3\pi}{5}$},{}},
grid=major,
ylabel={$|G| \axisunit{dB}$},
height=0.25\figureheight,
width=0.25\figurewidth,
axis x line*=bottom,
axis y line*=left]

% ================================================
\nextgroupplot[cheby1, legend columns=-1,
  legend cell align=left, legend style={at={(rel axis cs:0.5,1.05)}, anchor=south west}]
\node[plotnum,anchor=south west] at (rel axis cs:0,0) {1};
\addplot [color=G0Hat,solid,forget plot]
  table{\thisDir/fig/data/stress-frf/bodeplots-1.tsv};
\addplot [color=G0Hat,solid]
  table{\thisDir/fig/data/stress-frf/bodeplots-2.tsv};
\addlegendentry{\model{\trueSymbol}};
\label{leg:bode:model0}

\addplot [color=existing,solid,forget plot]
  table{\thisDir/fig/data/stress-frf/bodeplots-3.tsv};
\addplot [color=existing,solid]
  table{\thisDir/fig/data/stress-frf/bodeplots-4.tsv};
\addlegendentry{\model{exist}};
\label{leg:bode:modelExisting}

\addplot [color=LPMTrunc,solid,forget plot]
  table{\thisDir/fig/data/stress-frf/bodeplots-5.tsv};
\addplot [color=LPMTrunc,solid]
  table{\thisDir/fig/data/stress-frf/bodeplots-6.tsv};
  \addlegendentry{\model{trunc}};
\label{leg:bode:modelTrunc}

\addplot [color=RFIR,solid,forget plot]
  table{\thisDir/fig/data/stress-frf/bodeplots-7.tsv};
\addplot [color=RFIR,solid]
  table{\thisDir/fig/data/stress-frf/bodeplots-8.tsv};
\addlegendentry{\model{RFIR}};
\label{leg:bode:model:RFIR}

% \addplot [color=best,dotted]
%   table{\thisDir/fig/data/stress-frf/bodeplots-9.tsv};
% \addplot [color=best,dotted]
%   table{\thisDir/fig/data/stress-frf/bodeplots-10.tsv};
% \addlegendentry{\model{best}};
% \label{leg:bode:best}
\addplot [exact]
  table{\thisDir/fig/data/stress-frf/bodeplots-11.tsv};
\addlegendentry{$\true{G}$};
\label{leg:bode:true}


% ================================================
\nextgroupplot[cheby1]
\node[plotnum,anchor=south] at (rel axis cs:0.5,0) {2};
\addplot [color=G0Hat,solid]
  table{\thisDir/fig/data/stress-frf/bodeplots-12.tsv};
\addplot [color=G0Hat,solid]
  table{\thisDir/fig/data/stress-frf/bodeplots-13.tsv};
\addplot [color=existing,solid]
  table{\thisDir/fig/data/stress-frf/bodeplots-14.tsv};
\addplot [color=existing,solid]
  table{\thisDir/fig/data/stress-frf/bodeplots-15.tsv};
\addplot [color=LPMTrunc,solid]
  table{\thisDir/fig/data/stress-frf/bodeplots-16.tsv};
\addplot [color=LPMTrunc,solid]
  table{\thisDir/fig/data/stress-frf/bodeplots-17.tsv};
\addplot [color=RFIR,solid]
  table{\thisDir/fig/data/stress-frf/bodeplots-18.tsv};
\addplot [color=RFIR,solid]
  table{\thisDir/fig/data/stress-frf/bodeplots-19.tsv};
% \addplot [color=best,dotted]
%   table{\thisDir/fig/data/stress-frf/bodeplots-20.tsv};
% \addplot [color=best,dotted]
%   table{\thisDir/fig/data/stress-frf/bodeplots-21.tsv};
\addplot [exact]
  table{\thisDir/fig/data/stress-frf/bodeplots-22.tsv};

% ================================================
\nextgroupplot[cheby1]
\node[plotnum,anchor=south east] at (rel axis cs:1,0) {3};
\addplot [color=G0Hat,solid]
  table{\thisDir/fig/data/stress-frf/bodeplots-23.tsv};
\addplot [color=G0Hat,solid]
  table{\thisDir/fig/data/stress-frf/bodeplots-24.tsv};
\addplot [color=existing,solid]
  table{\thisDir/fig/data/stress-frf/bodeplots-25.tsv};
\addplot [color=existing,solid]
  table{\thisDir/fig/data/stress-frf/bodeplots-26.tsv};
\addplot [color=LPMTrunc,solid]
  table{\thisDir/fig/data/stress-frf/bodeplots-27.tsv};
\addplot [color=LPMTrunc,solid]
  table{\thisDir/fig/data/stress-frf/bodeplots-28.tsv};
\addplot [color=RFIR,solid]
  table{\thisDir/fig/data/stress-frf/bodeplots-29.tsv};
\addplot [color=RFIR,solid]
  table{\thisDir/fig/data/stress-frf/bodeplots-30.tsv};
% \addplot [color=best,dotted]
%   table{\thisDir/fig/data/stress-frf/bodeplots-31.tsv};
% \addplot [color=best,dotted]
%   table{\thisDir/fig/data/stress-frf/bodeplots-32.tsv};
\addplot [exact]
  table{\thisDir/fig/data/stress-frf/bodeplots-33.tsv};

% ================================================
\nextgroupplot[cheby1]
\node[plotnum,anchor=south east] at (rel axis cs:1,0) {4};
\addplot [color=G0Hat,solid]
  table{\thisDir/fig/data/stress-frf/bodeplots-34.tsv};
\addplot [color=G0Hat,solid]
  table{\thisDir/fig/data/stress-frf/bodeplots-35.tsv};
\addplot [color=existing,solid]
  table{\thisDir/fig/data/stress-frf/bodeplots-36.tsv};
\addplot [color=existing,solid]
  table{\thisDir/fig/data/stress-frf/bodeplots-37.tsv};
\addplot [color=LPMTrunc,solid]
  table{\thisDir/fig/data/stress-frf/bodeplots-38.tsv};
\addplot [color=LPMTrunc,solid]
  table{\thisDir/fig/data/stress-frf/bodeplots-39.tsv};
\addplot [color=RFIR,solid]
  table{\thisDir/fig/data/stress-frf/bodeplots-40.tsv};
\addplot [color=RFIR,solid]
  table{\thisDir/fig/data/stress-frf/bodeplots-41.tsv};
% \addplot [color=best,dotted]
%   table{\thisDir/fig/data/stress-frf/bodeplots-42.tsv};
% \addplot [color=best,dotted]
%   table{\thisDir/fig/data/stress-frf/bodeplots-43.tsv};
\addplot [exact]
  table{\thisDir/fig/data/stress-frf/bodeplots-44.tsv};

% ================================================
% ================================================
\nextgroupplot[cheby2]
\node[plotnum,anchor=south west] at (rel axis cs:0,0) {5};
\addplot [color=G0Hat,solid]
  table{\thisDir/fig/data/stress-frf/bodeplots-45.tsv};
\addplot [color=G0Hat,solid]
  table{\thisDir/fig/data/stress-frf/bodeplots-46.tsv};
\addplot [color=existing,solid]
  table{\thisDir/fig/data/stress-frf/bodeplots-47.tsv};
\addplot [color=existing,solid]
  table{\thisDir/fig/data/stress-frf/bodeplots-48.tsv};
\addplot [color=LPMTrunc,solid]
  table{\thisDir/fig/data/stress-frf/bodeplots-49.tsv};
\addplot [color=LPMTrunc,solid]
  table{\thisDir/fig/data/stress-frf/bodeplots-50.tsv};
\addplot [color=RFIR,solid]
  table{\thisDir/fig/data/stress-frf/bodeplots-51.tsv};
\addplot [color=RFIR,solid]
  table{\thisDir/fig/data/stress-frf/bodeplots-52.tsv};
% \addplot [color=best,dotted]
%   table{\thisDir/fig/data/stress-frf/bodeplots-53.tsv};
% \addplot [color=best,dotted]
%   table{\thisDir/fig/data/stress-frf/bodeplots-54.tsv};
\addplot [exact]
  table{\thisDir/fig/data/stress-frf/bodeplots-55.tsv};

% ================================================
\nextgroupplot[cheby2]
\node[plotnum,anchor=south] at (rel axis cs:0.5,0) {6};
\addplot [color=G0Hat,solid]
  table{\thisDir/fig/data/stress-frf/bodeplots-56.tsv};
\addplot [color=G0Hat,solid]
  table{\thisDir/fig/data/stress-frf/bodeplots-57.tsv};
\addplot [color=existing,solid]
  table{\thisDir/fig/data/stress-frf/bodeplots-58.tsv};
\addplot [color=existing,solid]
  table{\thisDir/fig/data/stress-frf/bodeplots-59.tsv};
\addplot [color=LPMTrunc,solid]
  table{\thisDir/fig/data/stress-frf/bodeplots-60.tsv};
\addplot [color=LPMTrunc,solid]
  table{\thisDir/fig/data/stress-frf/bodeplots-61.tsv};
\addplot [color=RFIR,solid]
  table{\thisDir/fig/data/stress-frf/bodeplots-62.tsv};
\addplot [color=RFIR,solid]
  table{\thisDir/fig/data/stress-frf/bodeplots-63.tsv};
% \addplot [color=best,dotted]
%   table{\thisDir/fig/data/stress-frf/bodeplots-64.tsv};
% \addplot [color=best,dotted]
%   table{\thisDir/fig/data/stress-frf/bodeplots-65.tsv};
\addplot [exact]
  table{\thisDir/fig/data/stress-frf/bodeplots-66.tsv};

% ================================================
\nextgroupplot[cheby2]
\node[plotnum,anchor=south east] at (rel axis cs:1,0) {7};
\addplot [color=G0Hat,solid]
  table{\thisDir/fig/data/stress-frf/bodeplots-67.tsv};
\addplot [color=G0Hat,solid]
  table{\thisDir/fig/data/stress-frf/bodeplots-68.tsv};
\addplot [color=existing,solid]
  table{\thisDir/fig/data/stress-frf/bodeplots-69.tsv};
\addplot [color=existing,solid]
  table{\thisDir/fig/data/stress-frf/bodeplots-70.tsv};
\addplot [color=LPMTrunc,solid]
  table{\thisDir/fig/data/stress-frf/bodeplots-71.tsv};
\addplot [color=LPMTrunc,solid]
  table{\thisDir/fig/data/stress-frf/bodeplots-72.tsv};
\addplot [color=RFIR,solid]
  table{\thisDir/fig/data/stress-frf/bodeplots-73.tsv};
\addplot [color=RFIR,solid]
  table{\thisDir/fig/data/stress-frf/bodeplots-74.tsv};
% \addplot [color=best,dotted]
%   table{\thisDir/fig/data/stress-frf/bodeplots-75.tsv};
% \addplot [color=best,dotted]
%   table{\thisDir/fig/data/stress-frf/bodeplots-76.tsv};
\addplot [exact]
  table{\thisDir/fig/data/stress-frf/bodeplots-77.tsv};

% ================================================
\nextgroupplot[cheby2]
\node[plotnum,anchor=south east] at (rel axis cs:1,0) {8};
\addplot [color=G0Hat,solid]
  table{\thisDir/fig/data/stress-frf/bodeplots-78.tsv};
\addplot [color=G0Hat,solid]
  table{\thisDir/fig/data/stress-frf/bodeplots-79.tsv};
\addplot [color=existing,solid]
  table{\thisDir/fig/data/stress-frf/bodeplots-80.tsv};
\addplot [color=existing,solid]
  table{\thisDir/fig/data/stress-frf/bodeplots-81.tsv};
\addplot [color=LPMTrunc,solid]
  table{\thisDir/fig/data/stress-frf/bodeplots-82.tsv};
\addplot [color=LPMTrunc,solid]
  table{\thisDir/fig/data/stress-frf/bodeplots-83.tsv};
\addplot [color=RFIR,solid]
  table{\thisDir/fig/data/stress-frf/bodeplots-84.tsv};
\addplot [color=RFIR,solid]
  table{\thisDir/fig/data/stress-frf/bodeplots-85.tsv};
% \addplot [color=best,dotted]
%   table{\thisDir/fig/data/stress-frf/bodeplots-86.tsv};
% \addplot [color=best,dotted]
%   table{\thisDir/fig/data/stress-frf/bodeplots-87.tsv};
\addplot [exact]
  table{\thisDir/fig/data/stress-frf/bodeplots-88.tsv};

% ================================================
% ================================================
\nextgroupplot[ellip]
\node[plotnum,anchor=south west] at (rel axis cs:0,0) {9};
\addplot [color=G0Hat,solid]
  table{\thisDir/fig/data/stress-frf/bodeplots-89.tsv};
\addplot [color=G0Hat,solid]
  table{\thisDir/fig/data/stress-frf/bodeplots-90.tsv};
\addplot [color=existing,solid]
  table{\thisDir/fig/data/stress-frf/bodeplots-91.tsv};
\addplot [color=existing,solid]
  table{\thisDir/fig/data/stress-frf/bodeplots-92.tsv};
\addplot [color=LPMTrunc,solid]
  table{\thisDir/fig/data/stress-frf/bodeplots-93.tsv};
\addplot [color=LPMTrunc,solid]
  table{\thisDir/fig/data/stress-frf/bodeplots-94.tsv};
\addplot [color=RFIR,solid]
  table{\thisDir/fig/data/stress-frf/bodeplots-95.tsv};
\addplot [color=RFIR,solid]
  table{\thisDir/fig/data/stress-frf/bodeplots-96.tsv};
% \addplot [color=best,dotted]
%   table{\thisDir/fig/data/stress-frf/bodeplots-97.tsv};
% \addplot [color=best,dotted]
%   table{\thisDir/fig/data/stress-frf/bodeplots-98.tsv};
\addplot [exact]
  table{\thisDir/fig/data/stress-frf/bodeplots-99.tsv};

% ================================================
\nextgroupplot[ellip]
\node[plotnum,anchor=south] at (rel axis cs:0.5,0) {10};
\addplot [color=G0Hat,solid]
  table{\thisDir/fig/data/stress-frf/bodeplots-100.tsv};
\addplot [color=G0Hat,solid]
  table{\thisDir/fig/data/stress-frf/bodeplots-101.tsv};
\addplot [color=existing,solid]
  table{\thisDir/fig/data/stress-frf/bodeplots-102.tsv};
\addplot [color=existing,solid]
  table{\thisDir/fig/data/stress-frf/bodeplots-103.tsv};
\addplot [color=LPMTrunc,solid]
  table{\thisDir/fig/data/stress-frf/bodeplots-104.tsv};
\addplot [color=LPMTrunc,solid]
  table{\thisDir/fig/data/stress-frf/bodeplots-105.tsv};
\addplot [color=RFIR,solid]
  table{\thisDir/fig/data/stress-frf/bodeplots-106.tsv};
\addplot [color=RFIR,solid]
  table{\thisDir/fig/data/stress-frf/bodeplots-107.tsv};
% \addplot [color=best,dotted]
%   table{\thisDir/fig/data/stress-frf/bodeplots-108.tsv};
% \addplot [color=best,dotted]
%   table{\thisDir/fig/data/stress-frf/bodeplots-109.tsv};
\addplot [exact]
  table{\thisDir/fig/data/stress-frf/bodeplots-110.tsv};

% ================================================
\nextgroupplot[ellip]
\node[plotnum,anchor=south east] at (rel axis cs:1,0) {11};
\addplot [color=G0Hat,solid]
  table{\thisDir/fig/data/stress-frf/bodeplots-111.tsv};
\addplot [color=G0Hat,solid]
  table{\thisDir/fig/data/stress-frf/bodeplots-112.tsv};
\addplot [color=existing,solid]
  table{\thisDir/fig/data/stress-frf/bodeplots-113.tsv};
\addplot [color=existing,solid]
  table{\thisDir/fig/data/stress-frf/bodeplots-114.tsv};
\addplot [color=LPMTrunc,solid]
  table{\thisDir/fig/data/stress-frf/bodeplots-115.tsv};
\addplot [color=LPMTrunc,solid]
  table{\thisDir/fig/data/stress-frf/bodeplots-116.tsv};
\addplot [color=RFIR,solid]
  table{\thisDir/fig/data/stress-frf/bodeplots-117.tsv};
\addplot [color=RFIR,solid]
  table{\thisDir/fig/data/stress-frf/bodeplots-118.tsv};
% \addplot [color=best,dotted]
%   table{\thisDir/fig/data/stress-frf/bodeplots-119.tsv};
% \addplot [color=best,dotted]
%   table{\thisDir/fig/data/stress-frf/bodeplots-120.tsv};
\addplot [exact]
  table{\thisDir/fig/data/stress-frf/bodeplots-121.tsv};

% ================================================
\nextgroupplot[ellip]
\node[plotnum,anchor=south east] at (rel axis cs:1,0) {12};
\addplot [color=G0Hat,solid]
  table{\thisDir/fig/data/stress-frf/bodeplots-122.tsv};
\addplot [color=G0Hat,solid]
  table{\thisDir/fig/data/stress-frf/bodeplots-123.tsv};
\addplot [color=existing,solid]
  table{\thisDir/fig/data/stress-frf/bodeplots-124.tsv};
\addplot [color=existing,solid]
  table{\thisDir/fig/data/stress-frf/bodeplots-125.tsv};
\addplot [color=LPMTrunc,solid]
  table{\thisDir/fig/data/stress-frf/bodeplots-126.tsv};
\addplot [color=LPMTrunc,solid]
  table{\thisDir/fig/data/stress-frf/bodeplots-127.tsv};
\addplot [color=RFIR,solid]
  table{\thisDir/fig/data/stress-frf/bodeplots-128.tsv};
\addplot [color=RFIR,solid]
  table{\thisDir/fig/data/stress-frf/bodeplots-129.tsv};
% \addplot [color=best,dotted]
%   table{\thisDir/fig/data/stress-frf/bodeplots-130.tsv};
% \addplot [color=best,dotted]
%   table{\thisDir/fig/data/stress-frf/bodeplots-131.tsv};
\addplot [exact]
  table{\thisDir/fig/data/stress-frf/bodeplots-132.tsv};


% ================================================
% ================================================
\nextgroupplot[butter]
\node[plotnum,anchor=south west] at (rel axis cs:0,0) {13};
\addplot [color=G0Hat,solid]
  table{\thisDir/fig/data/stress-frf/bodeplots-133.tsv};
\addplot [color=G0Hat,solid]
  table{\thisDir/fig/data/stress-frf/bodeplots-134.tsv};
\addplot [color=existing,solid]
  table{\thisDir/fig/data/stress-frf/bodeplots-135.tsv};
\addplot [color=existing,solid]
  table{\thisDir/fig/data/stress-frf/bodeplots-136.tsv};
\addplot [color=LPMTrunc,solid]
  table{\thisDir/fig/data/stress-frf/bodeplots-137.tsv};
\addplot [color=LPMTrunc,solid]
  table{\thisDir/fig/data/stress-frf/bodeplots-138.tsv};
\addplot [color=RFIR,solid]
  table{\thisDir/fig/data/stress-frf/bodeplots-139.tsv};
\addplot [color=RFIR,solid]
  table{\thisDir/fig/data/stress-frf/bodeplots-140.tsv};
% \addplot [color=best,dotted]
%   table{\thisDir/fig/data/stress-frf/bodeplots-141.tsv};
% \addplot [color=best,dotted]
%   table{\thisDir/fig/data/stress-frf/bodeplots-142.tsv};
\addplot [exact]
  table{\thisDir/fig/data/stress-frf/bodeplots-143.tsv};

% ================================================
\nextgroupplot[butter]
\node[plotnum,anchor=south] at (rel axis cs:0.5,0) {14};
\addplot [color=G0Hat,solid]
  table{\thisDir/fig/data/stress-frf/bodeplots-144.tsv};
\addplot [color=G0Hat,solid]
  table{\thisDir/fig/data/stress-frf/bodeplots-145.tsv};
\addplot [color=existing,solid]
  table{\thisDir/fig/data/stress-frf/bodeplots-146.tsv};
\addplot [color=existing,solid]
  table{\thisDir/fig/data/stress-frf/bodeplots-147.tsv};
\addplot [color=LPMTrunc,solid]
  table{\thisDir/fig/data/stress-frf/bodeplots-148.tsv};
\addplot [color=LPMTrunc,solid]
  table{\thisDir/fig/data/stress-frf/bodeplots-149.tsv};
\addplot [color=RFIR,solid]
  table{\thisDir/fig/data/stress-frf/bodeplots-150.tsv};
\addplot [color=RFIR,solid]
  table{\thisDir/fig/data/stress-frf/bodeplots-151.tsv};
% \addplot [color=best,dotted]
%   table{\thisDir/fig/data/stress-frf/bodeplots-152.tsv};
% \addplot [color=best,dotted]
%   table{\thisDir/fig/data/stress-frf/bodeplots-153.tsv};
\addplot [exact]
  table{\thisDir/fig/data/stress-frf/bodeplots-154.tsv};

% ================================================
\nextgroupplot[butter]
\node[plotnum,anchor=south east] at (rel axis cs:1,0) {15};
\addplot [color=G0Hat,solid]
  table{\thisDir/fig/data/stress-frf/bodeplots-155.tsv};
\addplot [color=G0Hat,solid]
  table{\thisDir/fig/data/stress-frf/bodeplots-156.tsv};
\addplot [color=existing,solid]
  table{\thisDir/fig/data/stress-frf/bodeplots-157.tsv};
\addplot [color=existing,solid]
  table{\thisDir/fig/data/stress-frf/bodeplots-158.tsv};
\addplot [color=LPMTrunc,solid]
  table{\thisDir/fig/data/stress-frf/bodeplots-159.tsv};
\addplot [color=LPMTrunc,solid]
  table{\thisDir/fig/data/stress-frf/bodeplots-160.tsv};
\addplot [color=RFIR,solid]
  table{\thisDir/fig/data/stress-frf/bodeplots-161.tsv};
\addplot [color=RFIR,solid]
  table{\thisDir/fig/data/stress-frf/bodeplots-162.tsv};
% \addplot [color=best,dotted]
%   table{\thisDir/fig/data/stress-frf/bodeplots-163.tsv};
% \addplot [color=best,dotted]
%   table{\thisDir/fig/data/stress-frf/bodeplots-164.tsv};
\addplot [exact]
  table{\thisDir/fig/data/stress-frf/bodeplots-165.tsv};

% ================================================
\nextgroupplot[butter]
\node[plotnum,anchor=south east] at (rel axis cs:1,0) {16};
\addplot [color=G0Hat,solid]
  table{\thisDir/fig/data/stress-frf/bodeplots-166.tsv};
\addplot [color=G0Hat,solid]
  table{\thisDir/fig/data/stress-frf/bodeplots-167.tsv};
\addplot [color=existing,solid]
  table{\thisDir/fig/data/stress-frf/bodeplots-168.tsv};
\addplot [color=existing,solid]
  table{\thisDir/fig/data/stress-frf/bodeplots-169.tsv};
\addplot [color=LPMTrunc,solid]
  table{\thisDir/fig/data/stress-frf/bodeplots-170.tsv};
\addplot [color=LPMTrunc,solid]
  table{\thisDir/fig/data/stress-frf/bodeplots-171.tsv};
\addplot [color=RFIR,solid]
  table{\thisDir/fig/data/stress-frf/bodeplots-172.tsv};
\addplot [color=RFIR,solid]
  table{\thisDir/fig/data/stress-frf/bodeplots-173.tsv};
% \addplot [color=best,dotted]
%   table{\thisDir/fig/data/stress-frf/bodeplots-174.tsv};
% \addplot [color=best,dotted]
%   table{\thisDir/fig/data/stress-frf/bodeplots-175.tsv};
\addplot [exact]
  table{\thisDir/fig/data/stress-frf/bodeplots-176.tsv};

\end{groupplot}

% \node at (bodes c4r3.north east) [xshift=1.5cm] {\pgfplotslegendfromname{legBodes}};

\end{tikzpicture}

  \caption{Bode plots of the tested filters~\legref{leg:bode:true}. Top to bottom: Chebyshev I, Chebyshev II, Elliptical and Butterworth characteristic.
  Left to right: low-pass, band-pass, high-pass and band-stop.
  For each filter, two of the parametric estimates $\model{\bullet}$ are shown per method.
  }
  \label{fig:bodeplots}
\end{figure}

\begin{figure}[p]
  \setlength{\figurewidth}{0.75\columnwidth}
  \setlength{\figureheight}{0.6\figurewidth}
  \centering
  % This file was created by matlab2tikz.
%
%workdir  : /Users/egon/Dropbox/VUB/PhD/Mikaya/initialvalues
%stack    : ../../../../../../../../Users/egon/Dropbox/VUB/PhD/Mikaya/initialvalues/testMonteCarloSystemsSmall (10)
%git info : commit 7bee89de2a914143a08c391cf0cbb846b1db5777
%           Author: Egon Geerardyn <egon.geerardyn@gmail.com>
%           Date:   Mon Feb 2 18:15:16 2015 +0100
%           
%               lo SNR stress test with Cheby
%           
%            printSummary.m               | 10 +++++----
%            summary.m                    |  2 ++
%            testMonteCarloSystemsSmall.m | 51 ++++++++++++++++++++++----------------------
%            3 files changed, 34 insertions(+), 29 deletions(-)
%           
%
%
%
\usepgfplotslibrary{groupplots}
\tikzset{plotnum/.style={color=black!30,align=left,font=\scriptsize,anchor=north west}}

\begin{tikzpicture}

\pgfplotsset{every axis legend/.style={anchor=west}}

\begin{groupplot}[%
group style={%
  group name=dist,
  group size=4 by 4,
  horizontal sep=0.2cm,
  vertical sep=0.2cm,
  x descriptions at=edge bottom,
  y descriptions at=edge left},
xlabel={Run},
scale only axis,
xmin=1,
grid=major,
ytick={1,0.1,0.01},
xtick={25,50,75},
extra x ticks={100},
extra x tick labels={},
extra y ticks={0.02},
extra y tick labels={},
xmax=100,
ylabel={$D_{\bullet}$},
ymin=0.0009,
ymax=1,
ymode=log,
height=0.25\figureheight,
width=0.25\figurewidth,
axis x line*=bottom,
axis y line*=left]

% ================================================
\nextgroupplot[%
  legend columns=-1,
  legend to name=leg:distances,
  legend cell align=left
]
\node[plotnum] at (rel axis cs:0,1) {1};
\addplot [color=G0Hat,solid]
  table{\thisDir/fig/distances-1.tsv};
\addlegendentry{\model{0}};
\label{leg:dist:model0}

\addplot [color=existing,solid]
  table{\thisDir/fig/distances-2.tsv};
\addlegendentry{\model{exist}};
\label{leg:dist:modelExist}

\addplot [color=LPMTrunc,solid]
  table{\thisDir/fig/distances-3.tsv};
\label{leg:dist:modelTrunc}
\addlegendentry{\model{trunc}};

\addplot [color=RFIR,solid]
  table{\thisDir/fig/distances-4.tsv};
\label{leg:dist:modelRFIR}
\addlegendentry{\model{RFIR}};

\addplot [color=best,dashed]
  table{\thisDir/fig/distances-5.tsv};
\label{leg:dist:modelBest}
\addlegendentry{\model{best}};

% ================================================
\nextgroupplot
\node[plotnum] at (rel axis cs:0,1) {2};
\addplot [color=G0Hat,solid,forget plot]
  table{\thisDir/fig/distances-6.tsv};
\addplot [color=existing,solid,forget plot]
  table{\thisDir/fig/distances-7.tsv};
\addplot [color=LPMTrunc,solid,forget plot]
  table{\thisDir/fig/distances-8.tsv};
\addplot [color=RFIR,solid,forget plot]
  table{\thisDir/fig/distances-9.tsv};
\addplot [color=best,dashed,forget plot]
  table{\thisDir/fig/distances-10.tsv};

% ================================================
\nextgroupplot
\node[plotnum] at (rel axis cs:0,1) {3};
\addplot [color=G0Hat,solid,forget plot]
  table{\thisDir/fig/distances-11.tsv};
\addplot [color=existing,solid,forget plot]
  table{\thisDir/fig/distances-12.tsv};
\addplot [color=LPMTrunc,solid,forget plot]
  table{\thisDir/fig/distances-13.tsv};
\addplot [color=RFIR,solid,forget plot]
  table{\thisDir/fig/distances-14.tsv};
\addplot [color=best,dashed,forget plot]
  table{\thisDir/fig/distances-15.tsv};

% ================================================
\nextgroupplot
\node[plotnum] at (rel axis cs:0,1) {4};
\addplot [color=G0Hat,solid,forget plot]
  table{\thisDir/fig/distances-16.tsv};
\addplot [color=existing,solid,forget plot]
  table{\thisDir/fig/distances-17.tsv};
\addplot [color=LPMTrunc,solid,forget plot]
  table{\thisDir/fig/distances-18.tsv};
\addplot [color=RFIR,solid,forget plot]
  table{\thisDir/fig/distances-19.tsv};
\addplot [color=best,dashed,forget plot]
  table{\thisDir/fig/distances-20.tsv};

% ================================================
% ================================================
\nextgroupplot
\node[plotnum] at (rel axis cs:0,1) {5};
\addplot [color=G0Hat,solid,forget plot]
  table{\thisDir/fig/distances-21.tsv};
\addplot [color=existing,solid,forget plot]
  table{\thisDir/fig/distances-22.tsv};
\addplot [color=LPMTrunc,solid,forget plot]
  table{\thisDir/fig/distances-23.tsv};
\addplot [color=RFIR,solid,forget plot]
  table{\thisDir/fig/distances-24.tsv};
\addplot [color=best,dashed,forget plot]
  table{\thisDir/fig/distances-25.tsv};

% ================================================
\nextgroupplot
\node[plotnum] at (rel axis cs:0,1) {6};
\addplot [color=G0Hat,solid,forget plot]
  table{\thisDir/fig/distances-26.tsv};
\addplot [color=existing,solid,forget plot]
  table{\thisDir/fig/distances-27.tsv};
\addplot [color=LPMTrunc,solid,forget plot]
  table{\thisDir/fig/distances-28.tsv};
\addplot [color=RFIR,solid,forget plot]
  table{\thisDir/fig/distances-29.tsv};
\addplot [color=best,dashed,forget plot]
  table{\thisDir/fig/distances-30.tsv};

% ================================================
\nextgroupplot
\node[plotnum] at (rel axis cs:0,1) {7};
\addplot [color=G0Hat,solid,forget plot]
  table{\thisDir/fig/distances-31.tsv};
\addplot [color=existing,solid,forget plot]
  table{\thisDir/fig/distances-32.tsv};
\addplot [color=LPMTrunc,solid,forget plot]
  table{\thisDir/fig/distances-33.tsv};
\addplot [color=RFIR,solid,forget plot]
  table{\thisDir/fig/distances-34.tsv};
\addplot [color=best,dashed,forget plot]
  table{\thisDir/fig/distances-35.tsv};

% ================================================
\nextgroupplot
\node[plotnum] at (rel axis cs:0,1) {8};
\addplot [color=G0Hat,solid,forget plot]
  table{\thisDir/fig/distances-36.tsv};
\addplot [color=existing,solid,forget plot]
  table{\thisDir/fig/distances-37.tsv};
\addplot [color=LPMTrunc,solid,forget plot]
  table{\thisDir/fig/distances-38.tsv};
\addplot [color=RFIR,solid,forget plot]
  table{\thisDir/fig/distances-39.tsv};
\addplot [color=best,dashed,forget plot]
  table{\thisDir/fig/distances-40.tsv};

% ================================================
% ================================================
\nextgroupplot
\node[plotnum] at (rel axis cs:0,1) {9};
\addplot [color=G0Hat,solid,forget plot]
  table{\thisDir/fig/distances-41.tsv};
\addplot [color=existing,solid,forget plot]
  table{\thisDir/fig/distances-42.tsv};
\addplot [color=LPMTrunc,solid,forget plot]
  table{\thisDir/fig/distances-43.tsv};
\addplot [color=RFIR,solid,forget plot]
  table{\thisDir/fig/distances-44.tsv};
\addplot [color=best,dashed,forget plot]
  table{\thisDir/fig/distances-45.tsv};

% ================================================
\nextgroupplot
\node[plotnum] at (rel axis cs:0,1) {10};
\addplot [color=G0Hat,solid,forget plot]
  table{\thisDir/fig/distances-46.tsv};
\addplot [color=existing,solid,forget plot]
  table{\thisDir/fig/distances-47.tsv};
\addplot [color=LPMTrunc,solid,forget plot]
  table{\thisDir/fig/distances-48.tsv};
\addplot [color=RFIR,solid,forget plot]
  table{\thisDir/fig/distances-49.tsv};
\addplot [color=best,dashed,forget plot]
  table{\thisDir/fig/distances-50.tsv};

% ================================================
\nextgroupplot
\node[plotnum] at (rel axis cs:0,1) {11};
\addplot [color=G0Hat,solid,forget plot]
  table{\thisDir/fig/distances-51.tsv};
\addplot [color=existing,solid,forget plot]
  table{\thisDir/fig/distances-52.tsv};
\addplot [color=LPMTrunc,solid,forget plot]
  table{\thisDir/fig/distances-53.tsv};
\addplot [color=RFIR,solid,forget plot]
  table{\thisDir/fig/distances-54.tsv};
\addplot [color=best,dashed,forget plot]
  table{\thisDir/fig/distances-55.tsv};

% ================================================
\nextgroupplot
\node[plotnum] at (rel axis cs:0,1) {12};
\addplot [color=G0Hat,solid,forget plot]
  table{\thisDir/fig/distances-56.tsv};
\addplot [color=existing,solid,forget plot]
  table{\thisDir/fig/distances-57.tsv};
\addplot [color=LPMTrunc,solid,forget plot]
  table{\thisDir/fig/distances-58.tsv};
\addplot [color=RFIR,solid,forget plot]
  table{\thisDir/fig/distances-59.tsv};
\addplot [color=best,dashed,forget plot]
  table{\thisDir/fig/distances-60.tsv};


% ================================================
% ================================================
\nextgroupplot
\node[plotnum] at (rel axis cs:0,1) {13};
\addplot [color=G0Hat,solid,forget plot]
  table{\thisDir/fig/distances-61.tsv};
\addplot [color=existing,solid,forget plot]
  table{\thisDir/fig/distances-62.tsv};
\addplot [color=LPMTrunc,solid,forget plot]
  table{\thisDir/fig/distances-63.tsv};
\addplot [color=RFIR,solid,forget plot]
  table{\thisDir/fig/distances-64.tsv};
\addplot [color=best,dashed,forget plot]
  table{\thisDir/fig/distances-65.tsv};

% ================================================
\nextgroupplot
\node[plotnum] at (rel axis cs:0,1) {14};
\addplot [color=G0Hat,solid,forget plot]
  table{\thisDir/fig/distances-66.tsv};
\addplot [color=existing,solid,forget plot]
  table{\thisDir/fig/distances-67.tsv};
\addplot [color=LPMTrunc,solid,forget plot]
  table{\thisDir/fig/distances-68.tsv};
\addplot [color=RFIR,solid,forget plot]
  table{\thisDir/fig/distances-69.tsv};
\addplot [color=best,dashed,forget plot]
  table{\thisDir/fig/distances-70.tsv};

% ================================================
\nextgroupplot
\node[plotnum] at (rel axis cs:0,1) {15};
\addplot [color=G0Hat,solid,forget plot]
  table{\thisDir/fig/distances-71.tsv};
\addplot [color=existing,solid,forget plot]
  table{\thisDir/fig/distances-72.tsv};
\addplot [color=LPMTrunc,solid,forget plot]
  table{\thisDir/fig/distances-73.tsv};
\addplot [color=RFIR,solid,forget plot]
  table{\thisDir/fig/distances-74.tsv};
\addplot [color=best,dashed,forget plot]
  table{\thisDir/fig/distances-75.tsv};

% ================================================
\nextgroupplot
\node[plotnum] at (rel axis cs:0,1) {16};
\addplot [color=G0Hat,solid,forget plot]
  table{\thisDir/fig/distances-76.tsv};
\addplot [color=existing,solid,forget plot]
  table{\thisDir/fig/distances-77.tsv};
\addplot [color=LPMTrunc,solid,forget plot]
  table{\thisDir/fig/distances-78.tsv};
\addplot [color=RFIR,solid,forget plot]
  table{\thisDir/fig/distances-79.tsv};
\addplot [color=best,dashed,forget plot]
  table{\thisDir/fig/distances-80.tsv};

\end{groupplot}

%\node at (dist c4r3.north east) [xshift=1.5cm] {\ref{legDistances}};

\end{tikzpicture}%

  \caption{Empirical cumulative distribution function of the validation distance $D_{\bullet} = \norm{\model{\bullet} - G_0}$ for the different tested systems (different numbered plots) and the different initial values (colors).
  Note that since $\norm{G_0}=1$ is chosen, $D_{\bullet}$ immediately indicates the relative RMS error of the estimates.
  }
  \label{fig:distancesStress}
\end{figure}

The transfer functions are shown in \figref{fig:bodeplots} while the empirical cumulative validation distance is shown in \figref{fig:distancesStress}.

Our general conclusion from these simulations is that RFIR performs as well (or a lot better) than the existing techniques when good models are observed.

{\footnotesize
A few additional remarks are appropriate with respect to these results:
\begin{itemize}
  \item In many of the tested situations (subplots 1,3,13-16), all methods yield a good model quality.
  \item In some situations (subplots 8, 10, 12), the model quality is generally poor. 
  The fact that for subplots 8 and 12, the `ideal' starting value \model{0} also provides poor models means the data is not informative enough to reasonably fit a good model.
  \item In all cases except 8, 10 and 12, the RFIR performed as well or better than the existing methods.
  \item In subplot 10, the model quality of the RFIR is slightly worse than the already poor existing techniques.
  This is a underlying limitation of the RFIR used: $n_g=200$ taps was used, however the true impulse response has not decayed significantly in that interval.
  Essentially, the particular RFIR used introduces a significant bias in the non-parametric description of the filter and hence the obtained initial value is unreliable.
  \item The RFIR performs (significantly) better than the existing methods in various cases (subplot 4-7, 9, 11).
  \item Since the `best' method \legref{leg:dist:modelBest} yields better models more often than a single method, e.g. in subplots 2, 4 and 6, this illustrates that combining initialization schemes improves the model quality.
  Particularly, subplot 2 implies that \model{best} contains initial estimates of all different techniques.
  \item For all of the tested systems (except perhaps subplot 2), the truncated LPM is not advisable over the existing methods.
\end{itemize}
}
