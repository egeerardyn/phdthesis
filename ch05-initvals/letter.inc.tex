
\newcommand{\reviewer}[1]{\paragraph{#1}}

In \figref{fig:overview} we can compare $\model{2nd}$ and $\model{best}$.
By inspecting the different estimates for  each of the repeated experiments (\figref{fig:overview}), it can easily be seen that choosing $\model{RFIR}$ while it is not $\model{best}$ only leads to a very modest performance degradation.
In \figref{fig:histogram}, the different methods are compared in this respect by subtracting the performance of the $\model{best}$ from the performance of the respective methods.
From that figure it can be seen that there is little performance to gain \emph{or} lose from choosing $\model{RFIR}$ over $\model{best}$ when RFIR is not the `best' method (as measured by the cost function $V$).
The outcome is a lot less favorable for the alternative methods.

\begin{figure}
  \centering
  \setlength{\figurewidth}{0.85\columnwidth}
  \setlength{\figureheight}{0.68\figurewidth}
  % This file was created by matlab2tikz.
%
%workdir  : /Users/egon/Dropbox/VUB/PhD/Mikaya/initialvalues
%stack    : ../../../../../../../../Users/egon/Dropbox/VUB/PhD/Mikaya/initialvalues/processAllMeasurements (10)
%git info : commit 45c2d89a4e4044db9de2834c5c1c86bec69e4864
%           Author: Egon Geerardyn <egon.geerardyn@gmail.com>
%           Date:   Tue Oct 7 09:18:45 2014 +0200
%           
%               estimParamMeas: duplicate code removed
%           
%            estimateParametricMeas.m | 9 +--------
%            1 file changed, 1 insertion(+), 8 deletions(-)
%           
%
%
\begin{tikzpicture}
\begin{axis}[%
name=fit,
width=\figurewidth,
height=0.5\figureheight,
scale only axis,
xmin=1,
xmax=100,
xmajorgrids,
xtick={15,23,41,53,69,88,100},
xticklabels={},
ymode=log,
ymin=4000,
ymax=100000,
yminorticks=true,
ylabel={Cost $V(\model{\bullet})$},
ymajorgrids,
yminorgrids,
axis x line*=bottom,
axis y line*=left,
]
\addplot [best,line width=1pt] table[]{\thisDir/data/overview-cost-val/cost-best.tsv};
\label{leg:best}
% \addlegendentry{$\hat{G}_{best}$}

\addplot [RFIR,RFIRmark,only marks] table[]{\thisDir/data/overview-cost-val/cost-rfir.tsv};
\label{leg:RFIR}
% \addlegendentry{$\hat{G}_{RFIR}$}

\addplot [LPMTrunc,LPMTruncInitmark,only marks] table[]{\thisDir/data/overview-cost-val/cost-trunc.tsv};
\label{leg:trunc}
% \addlegendentry{$\hat{G}_{trunc}$}

\addplot [existing,existingInitmark,only marks] table[]{\thisDir/data/overview-cost-val/cost-exist.tsv};
\label{leg:exist}
% \addlegendentry{$\hat{G}_{existing}$}

\addplot [G0Hat,G0Hatmark,only marks,medsmallmarkers] table[]{\thisDir/data/overview-cost-val/cost-true.tsv};
% \addlegendentry{$\hat{G}_0$}
\label{leg:VXI}

\end{axis}


\begin{axis}[%
name=valid,
width=\figurewidth,
height=0.5\figureheight,
anchor=above north west,
at={(fit.below south west)},
scale only axis,
xmin=1,
xmax=100,
xmajorgrids,
xtick={15,23,41,53,69,88,100},
ymode=log,
ymin=0.4,
ymax=90,
yminorticks=true,
xlabel={Repetition},
ylabel={$\norm[2]{\model{\bullet} - \model{VXI}}$},
ymajorgrids,
yminorgrids,
axis x line*=bottom,
axis y line*=left
]
\addplot [dashed] table[]{\thisDir/data/overview-cost-val/sys-h2norm.tsv};
\label{leg:reference}
\addplot [best,line width=1pt] table[]{\thisDir/data/overview-cost-val/distance-best.tsv};
\addplot [RFIR,RFIRmark,only marks] table[]{\thisDir/data/overview-cost-val/distance-rfir.tsv};
\addplot [LPMTrunc,LPMTruncInitmark,only marks] table[]{\thisDir/data/overview-cost-val/distance-trunc.tsv};
\addplot [existing,existingInitmark,only marks] table[]{\thisDir/data/overview-cost-val/distance-exist.tsv};
\addplot [G0Hat,G0Hatmark,only marks,medsmallmarkers] table[]{\thisDir/data/overview-cost-val/distance-true.tsv};
\end{axis}
\end{tikzpicture}%

  \caption[$\costFunc{\bullet}$ and $\validationDistance{\bullet}$ for each repeated measurement.]{Cost function $V(\model{\bullet})$ and validation distance of the model estimated in each repetition of the measurement.
  Based on the cost function, $\model{RFIR}$~\legref{leg:RFIR} is not always the best estimate $\model{best}$~\legref{leg:best}.
  Especially in the validation plot (bottom), $\model{RFIR}$ performs (almost) as well or even better than $\model{exist}$~\legref{leg:exist} and $\model{trunc}$~\legref{leg:trunc}.
  Both latter methods often produce models that perform poorly compared to $\| G_{VXI} \|$~\legref{leg:reference} and $\model{VXI}$~\legref{leg:VXI}.
  This means that in our limited experimental study, the performance degradation of choosing RFIR to produce initial values over the studied alternatives is negligible.}
  \label{fig:overview}
\end{figure}

\begin{figure}
  \centering
  \ref{leg:init:secondBest}
  \setlength{\figurewidth}{0.75\columnwidth}
  \setlength{\figureheight}{0.60\figurewidth}
  % This file was created by matlab2tikz.
%
%workdir  : /Users/egon/Dropbox/VUB/PhD/Mikaya/initialvalues
%stack    : ../../../../../../../../Users/egon/Dropbox/VUB/PhD/Mikaya/initialvalues/processAllMeasurements (10)
%git info : commit 45c2d89a4e4044db9de2834c5c1c86bec69e4864
%           Author: Egon Geerardyn <egon.geerardyn@gmail.com>
%           Date:   Tue Oct 7 09:18:45 2014 +0200
%
%               estimParamMeas: duplicate code removed
%
%            estimateParametricMeas.m | 9 +--------
%            1 file changed, 1 insertion(+), 8 deletions(-)
%
%
%
\begin{tikzpicture}
\pgfplotsset{every axis title/.append style={font={\footnotesize},at={(rel axis cs:0.5,0.97),anchor=center}}}
\begin{axis}[%
name=Estim,
width=0.5\figurewidth,
height=\figureheight,
scale only axis,
unbounded coords=jump,
xmin=0.5,
xmax=3.5,
grid=both,
title={Estimation: $V_{\bullet} = V(\model{\bullet})$ $\vphantom{\norm[2]{\model{\bullet}}}$},
x tick label as interval,
xtick={0.5,1.5,2.5,3.5},
xticklabels={{$\model{exist}$},{$\model{trunc}$},{$\model{RFIR}$}},
ymode=log,
ymin=0.001,
ymax=100000,
yminorticks=true,
ylabel={$V_{\bullet} - V_{\mathrm{best}}$},
axis background/.style={fill=white},
axis x line*=bottom,
axis y line*=left
]

\addplot[degradationBox, forget plot] table[] {\thisDir/data/special-histogram/estim-exist-degradation-box.tsv}--cycle;
\addplot [degradationMedian, forget plot] table[]{\thisDir/data/special-histogram/estim-exist-degradation-median.tsv};
\addplot [degradationMarks, forget plot] table[]{\thisDir/data/special-histogram/estim-exist-degradation-data.tsv};
\addplot[degradationBox, forget plot] table[] {\thisDir/data/special-histogram/estim-trunc-degradation-box.tsv}--cycle;
\addplot [degradationMedian, forget plot] table[]{\thisDir/data/special-histogram/estim-trunc-degradation-median.tsv};
\addplot [degradationMarks, forget plot] table[]{\thisDir/data/special-histogram/estim-trunc-degradation-data.tsv};

\addplot[degradationBox, forget plot] table[] {\thisDir/data/special-histogram/estim-RFIR-degradation-box.tsv}--cycle;
\addplot [degradationMedian, forget plot] table[]{\thisDir/data/special-histogram/estim-RFIR-degradation-median.tsv};
\addplot [degradationMarks, forget plot] table[]{\thisDir/data/special-histogram/estim-RFIR-degradation-data.tsv};

\addplot [percentileLines, forget plot] table[]{\thisDir/data/special-histogram/estim-max.tsv};
\addplot [percentileLines, forget plot] table[]{\thisDir/data/special-histogram/estim-median.tsv};
\addplot [percentileLines, forget plot] table[]{\thisDir/data/special-histogram/estim-min.tsv};
\end{axis}

\begin{axis}[%
name=Valid,
anchor=north west,
at={($(Estim.north east) + (1em,0pt)$)},
width=0.5\figurewidth,
height=\figureheight,
scale only axis,
unbounded coords=jump,
xmin=0.5,
xmax=3.5,
grid=both,
x tick label as interval,
xtick={0.5,1.5,2.5,3.5},
xticklabels={{$\model{exist}$},{$\model{trunc}$},{$\model{RFIR}$}},
ymode=log,
ymin=0.0001,
ymax=100,
yminorticks=true,
ylabel={${D_{\bullet} - D}_{\mathrm{best}}$},
title={Validation: $D_{\bullet} = \norm[2]{\model{\bullet} - G_{\mathrm{VXI}}}$},
axis x line*=bottom,
axis y line*=right,
% legend to name=leg:init:secondBest,
legend columns=-1,
legend cell align=center,
legend style={draw=none, at={(rel axis cs:-0.05,1.1)},anchor=south}
]

\addplot[enhancementBox, forget plot] table[] {\thisDir/data/special-histogram/valid-exist-enhancement-box.tsv}--cycle;
\addplot [enhancementMedian, forget plot] table[]{\thisDir/data/special-histogram/valid-exist-enhancement-median.tsv};
\addplot [enhancementMarks, forget plot] table[]{\thisDir/data/special-histogram/valid-exist-enhancement-data.tsv};

\addplot[enhancementBox, forget plot] table[] {\thisDir/data/special-histogram/valid-trunc-enhancement-box.tsv}--cycle;
\addplot [enhancementMedian, forget plot] table[]{\thisDir/data/special-histogram/valid-trunc-enhancement-median.tsv};
\addplot [enhancementMarks, forget plot] table[]{\thisDir/data/special-histogram/valid-trunc-enhancement-data.tsv};

\addplot[enhancementBox] table[] {\thisDir/data/special-histogram/valid-RFIR-enhancement-box.tsv}--cycle;
%\addlegendentry{boxplot $X_{\bullet} < X_{\mathrm{best}}$}
\addlegendentry{and}

\addplot [enhancementMedian, forget plot] table[]{\thisDir/data/special-histogram/valid-RFIR-enhancement-median.tsv};
% \addlegendentry{$X_{\bullet} < X_{\mathrm{best}}$}

\addplot [enhancementMarks] table[]{\thisDir/data/special-histogram/valid-RFIR-enhancement-data.tsv};
\addlegendentry{$X_{\bullet} < X_{\mathrm{best}}$}

\addlegendimage{empty legend}
\addlegendentry{\vspace{2em}}

\addplot[degradationBox, forget plot] table[] {\thisDir/data/special-histogram/valid-exist-degradation-box.tsv}--cycle;
\addplot [degradationMedian, forget plot] table[]{\thisDir/data/special-histogram/valid-exist-degradation-median.tsv};
\addplot [degradationMarks, forget plot] table[]{\thisDir/data/special-histogram/valid-exist-degradation-data.tsv};

\addplot[degradationBox, forget plot] table[]{\thisDir/data/special-histogram/valid-trunc-degradation-box.tsv}--cycle;
\addplot [degradationMedian, forget plot] table[]{\thisDir/data/special-histogram/valid-trunc-degradation-median.tsv};
\addplot [degradationMarks, forget plot] table[]{\thisDir/data/special-histogram/valid-trunc-degradation-data.tsv};

\addplot[degradationBox] table[] {\thisDir/data/special-histogram/valid-RFIR-degradation-box.tsv}--cycle;
% \addlegendentry{boxplot $X_{\bullet} > X_{\mathrm{best}}$}
\addlegendentry{and}

\addplot [degradationMedian, forget plot]  table[]{\thisDir/data/special-histogram/valid-RFIR-degradation-median.tsv};
% \addlegendentry{$X_{\bullet} > X_{\mathrm{best}}$}

\addplot [degradationMarks]  table[]{\thisDir/data/special-histogram/valid-RFIR-degradation-data.tsv};
\addlegendentry{$X_{\bullet} > X_{\mathrm{best}}$}

\addlegendimage{empty legend}
\addlegendentry{\vspace{2em}}

\addplot [percentileLines, forget plot]  table[]{\thisDir/data/special-histogram/valid-max.tsv};
\addplot [percentileLines, forget plot]  table[]{\thisDir/data/special-histogram/valid-median.tsv};
\addplot [percentileLines]   table[]{\thisDir/data/special-histogram/valid-min.tsv};
\addlegendentry{$\min, \mathrm{median}, \max \; X_{\bullet}$}

\end{axis}
\end{tikzpicture}%

  \caption[Performance degradation/enhancement for selecting the second best model.]{
  Performance degradation/enhancement ($X_{\bullet}-X_{\mathrm{best}}$) for choosing a given method $\bullet$ instead of the `best' method for the measurement example in the paper.
  In red/orange the positive values ($X_{\bullet} > X_{\mathrm{best}}$), the performance degradation for using $\model{\bullet}$ over $\model{best}$ is shown.
  The negative values ($X_{\bullet} < X_{\mathrm{best}}$), i.e. the performance enhancement (for using $\model{\bullet}$ instead of $\model{best}$) is shown in green/blue.
  Zero values (i.e. $X_{\bullet} = X_{\mathrm{best}}$) are not shown, hence the data shows when method $\bullet$ is \emph{not} selected as the `best' method.
  The dashed lines are the minimum, median and maximum of $X_{\bullet}$ to give a sense of scale.
  The performance degradation of choosing $\model{RFIR}$ instead of $\model{best}$ is smaller than the corresponding performance degradation for the other methods.
  The performance enhancement from choosing $\model{RFIR}$ over $\model{best}$, is almost on par with the corresponding performance degradation (i.e. there is little to gain \emph{or} lose from choosing $\model{RFIR}$ over $\model{best}$).
  For $\model{trunc}$ and $\model{exist}$, there is much less performance to gain from picking those over $\model{best}$ and there is a lot more to lose.
  }
  \label{fig:histogram}
\end{figure}

\begin{quote}

The use of a fixed method would greatly simplify an algorithm design in any standalone autonomous toolbox. The results in Table II do not help to uncover whether such a strategy is permissible. To solve this dilemma, two histograms of differences $\Delta = \model{best} - \model{2nd}$ need to be built and shown: one for $\model{best} = \model{FRIR}$ and the other for $\model{best} \neq \model{FRIR}$.
\end{quote}

We agree that a single method would greatly simplify the implementation of a toolbox.
However, we think that using multiple initial value strategies is a valuable approach since this makes a toolbox more robust to different conditions (SNR, frequency resolution, system order, \ldots) as it reduces the likelihood that a local optimum is hit.
This can e.g. be seen in subplot 2 of \figref{fig:distancesStress} (which is discussed further on).
A similar approach w.r.t. to combine initial values is followed in some toolboxes, e.g. \cite{FDIDENT,TDIDENT}.

Many nature-inspired optimization algorithms (such as Particle Swarm Optimization (PSO)) and stochastic optimization algorithms even use hundreds or thousands of randomized initial values, in part to avoid local optima.
But, obviously, using many initial values comes at a (considerable) increase in computational cost.

While this discussion and this inherent trade-off is very interesting, we think this is outside of the scope of the paper where we only wish to demonstrate that smoothing techniques such as RFIR are a worthwhile source of initial values, but not necessarily the only one.
